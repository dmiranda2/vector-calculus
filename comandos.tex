\usepackage{bookman} 
\usepackage[T1]{fontenc}
\usepackage{multicol}
\usepackage{answers}
\usepackage{makeidx}
\usepackage{commath}
% \usepackage{fancyhdr}
\usepackage{answers}
\usepackage{makeidx}
% \usepackage{avant} % Use the Avantgarde font for headings
% \usepackage{helvet}
\usepackage{microtype} % Slightly tweak font spacing for aesthetics
\usepackage[T1]{fontenc} %
\usepackage{pifont,esint}
\usepackage{pstricks}
\usepackage[tiling]{pst-fill}      % PSTricks package for filling/tiling
%%%%% COMANDOS %%%%%%%%%%%%%%%%%%%%%%%%%%%%%%%%%%%%%%%%%%%%%%%%%%
\newcommand{\dis}{\displaystyle }
\newcommand{\nt}{\noindent}
\definecolor{ocre}{cmyk}{0, 0.87, 0.68, 0.32} %Maroon
\definecolor{Maroon}{RGB}{30,144,255}
\definecolor{azul}{RGB}{30,144,255}
\definecolor{azulescuro}{RGB}{0.252, 0.419,0.584}
\definecolor{verde}{RGB}{127, 137, 97}
\definecolor{vermelho}{cmyk}{0, 0.87, 0.68, 0.32}
\definecolor{laranja}{RGB}{243,102,25}
\definecolor{blue}{rgb}{0.252, 0.419,0.584}
\definecolor{red}{cmyk}{0, 0.87, 0.68, 0.32}
\definecolor{amarelinho}{RGB}{255,249,215}
\definecolor{azulzinho}{RGB}{238,247,250}
%% Conjuntos
\newcommand{\bbR}{\ensuremath{\mathbb{R}}}
\newcommand{\IR}{I\kern -0.37 em R}
\newcommand{\bbH}{\mathbb{H}}
\newcommand{\IH}{I\kern -0.37 em H}
\newcommand{\bbL}{\mathbb{L}}
\newcommand{\IL}{I\kern -0.37 em L}
\newcommand{\bbN}{\mathbb{N}}
\newcommand{\bbZ}{\mathbb{Z}}
\newcommand{\bbQ}{\mathbb{Q}}
\newcommand{\bbC}{\mathbb{C}}
\newcommand{\bbS}{\mathbb{S}}
\newcommand{\bbE}{\mathbb{E}}
\newcommand{\bbU}{\mathbb{U}}
\newcommand{\0}{\varnothing}
%% Funes, operadores, álgebra linear, grupos, cálculo
\DeclareMathOperator{\Dom}{Dom}
\DeclareMathOperator{\Img}{Im}
\DeclareMathOperator{\Grafico}{Graf}
\renewcommand{\ker}{\mathrm{ker}\,}
\renewcommand{\dim}{\mathrm{dim}\,}
\newcommand{\End}{\mathrm{End}\,}
\newcommand{\Aut}{\mathrm{Aut}\,}
\newcommand{\Iso}{\mathrm{Iso}\,}
\providecommand{\Abs}[1]{\bigl\lvert\mspace{1mu}#1\mspace{1mu}\bigr\rvert}
\newcommand{\norma}[1]{\lVert#1\rVert}
\newcommand{\presc}[2]{\langle #1, #2 \rangle}
\newcommand{\vet}[1]{\overrightarrow{#1}}
\newcommand{\proj}[2]{\mathop{proj}_{\displaystyle #2} \,#1}
\DeclareMathOperator{\grad}{grad}
\DeclareMathOperator{\Hess}{Hess}
\DeclareMathOperator{\hess}{hess}
\newcommand{\negrito}[1]{\textbf{#1}\index{#1}}
\newcommand{\dx}{\ \dif x}
\newcommand{\du}{\ \dif u}
\newcommand{\dt}{\ \dif t}
\newcommand{\dv}{\ \dif v}
\newcommand{\dy}{\ \dif y}
\newcommand{\dz}{\ \dif z}
\newcommand{\ds}{\ \dif s}
\newcommand{\dsum}{\displaystyle\sum}
\newcommand{\barra}{\bigg|}
\renewcommand{\P}{\mathbb{P}}
%%%%%%%%%%%%%%restrição
\newcommand\restr[2]{\ensuremath{\left.#1\right|_{#2}}}
\newcommand{\qedsymbol}{{\tiny\ensuremath{\blacksquare}}}
% \newcommand{\comentario}[1]{\text{ \color{azul} \scriptsize #1}}
\newcommand{\supfi}{\max_{x\in I_i}(f(x))}
\newcommand{\inffi}{\min_{x\in I_i}(f(x))}
\newcommand{\Dxi}{\Delta x_i}
\newcommand{\sumi}{\sum_{i=1}^n}
%%%%%%%%%%%%%%%%%%%%%Cores
\newcommand{\vermelho}{\text{vermelho}}
\newcommand{\azul}{\text{azul}}
\renewcommand{\fbox}{\fcolorbox{white}{azulzinho}}
\newcommand{\destaque}[1]{\ensuremath{\fbox{\ensuremath{ #1 }}}}
%%%%%%%%%%%%%%%%%%%%%%%%%%%%%%%%%%%%%%
% math symbols
\def\lcrc#1#2{\left[#1 \ ; #2 \right]}
\def\dim{\mathop{\mathrm dim}\nolimits}
\def\EVAL{\mathop{\mathrm EVAL}\nolimits}
\def\SYM{\mathop{\mathrm SYM}\nolimits}
\def\ALT{\mathop{\mathrm ALT}\nolimits}
\def\Id{\mathop{\mathrm Id}\nolimits}
\def\Tr{\mathop{\mathrm Tr}\nolimits}
\def\Tan{{\mathrm T}}
\def\div{\mathop{\mathrm div}\nolimits}
\def\grad{\mathop{\mathrm grad}\nolimits}
\def\curl{\mathop{\mathrm curl}\nolimits}
\def\arcsinh{\mathop{\mathrm arcsinh}\nolimits}
\def\Lie{\mathop{\pounds}\nolimits}
\def\csch{\mathop{\mathrm csch}\nolimits}
\def\sech{\mathop{\mathrm csch}\nolimits}
\def\arccosh{\mathop{\mathrm arccosh}\nolimits}
\def\arccot{\mathop{\mathrm arccot}\nolimits}
\def\ad{\mathop{\mathrm ad}\nolimits}
\def\AD{\mathop{\mathrm AD}\nolimits}
\def\Ad{\mathop{\mathrm Ad}\nolimits}
\def\diag{\mathop{\mathrm diag}\nolimits}
\newcommand{\Dh}[2]{\frac{\partial #1}{\partial #2}} 

\providecommand{\Norm}[1]{\bigl\lVert\mspace{1mu}#1\mspace{1mu}\bigr\rVert}
\providecommand{\NORM}[1]{\Biggl\lVert\mspace{1mu}#1\mspace{1mu}\Biggr\rVert}
\providecommand{\ssub}[2]{#1_{\scriptscriptstyle #2}}
\providecommand{\Real}[1]{\mathbb{R}^{#1}}
\providecommand{\Dotprod}[2]{#1 \bp  #2}
\providecommand{\Crossprod}[2]{#1 \mathbf{\times} #2}
\providecommand{\Degrees}[0]{\ensuremath{^\circ}}
\providecommand{\ival}[2]{\lbrack #1,#2 \rbrack}
\providecommand{\lineintvec}[3]{\displaystyle\int_{#1} \Dotprod{\vector{#2}}{d\vector{#3}}}
\providecommand{\olineintvec}[3]{\displaystyle\oint_{#1} \Dotprod{\vector{#2}}{d\vector{#3}}}

\newcommand{\statedefn}[2]{
 \definecolor{shadethmcolor}{cmyk}{0.1,0.05,0,0}
 \definecolor{shaderulecolor}{cmyk}{0.73,0.19,0,0}
 \begin{definition}\label{#1}{#2}\end{definition}
}
\newcommand{\statethm}[2]{
 \definecolor{shadethmcolor}{HTML}{E8DCFF}
 \definecolor{shaderulecolor}{HTML}{9B5900}
 \begin{thm}\label{#1}{#2}\end{thm}
}
\newcommand{\statecor}[2]{
 \definecolor{shadethmcolor}{HTML}{FFFF8B}
 \definecolor{shaderulecolor}{HTML}{9B5900}
 \begin{cor}\label{#1}{#2}\end{cor}
}
\newcommand{\statecomment}[2][0.96\textwidth]{
 \par\noindent\tikzstyle{mybox} = [draw=gray!70!black,fill=gray!2,thick,rectangle,inner ysep=4pt]
 \begin{tikzpicture}
  \node [mybox] (box){%
   \begin{minipage}{#1}{#2}\end{minipage}
  };
 \end{tikzpicture}
}
\newcommand\myclearpage{\cleartooddpage
 [\thispagestyle{empty}]}


\renewcommand\sfdefault{phv}
\newcommand{\startexercises}{
%  \begin{center}\rule{\textwidth}{0.5pt}\end{center}
\section*{\psframebox{Exercises}}
}
\newcommand{\probs}[1]{\par\noindent\textsf{\textbf{\large #1}}}

\renewcommand{\arraystretch}{1.3}
\renewcommand{\d}{\dif}

 
\makeatletter
\newenvironment{tablehere}
  {\def\@captype{table}}
  {}

\newenvironment{figurehere}
  {\def\@captype{figure}}
  {}
\makeatother

\allowdisplaybreaks[1]
\newcommand{\inv}{^{-1}}
\DeclareMathOperator{\vol}{vol}
\DeclareMathOperator{\arclen}{length}
\newcommand{\defeq}{\stackrel{\scriptscriptstyle\text{def}}{=}}
\newcommand{\given}[1][\big]{\;#1|\;}
\newcommand{\st}[1][\big]{\given[#1]}
\newcommand{\lap}{\Delta}
\newcommand{\laplacian}{\lap}
\newcommand{\transpose}{^*}
\renewcommand{\grad}{\nabla}
\newcommand{\gradt}{\grad\transpose}
\newcommand{\gradperp}{\grad^\perp}
\newcommand{\curlb}{\grad \times}
\newcommand{\loc}{_\text{loc}}
\newcommand{\varmin}{\wedge}
\newcommand{\varmax}{\vee}
\newcommand{\as}{\text{a.s.}}
\newcommand{\E}{\mathbf{E}}
\renewcommand{\P}{\mathbf{P}}
\newcommand{\mc}{\mathcal}
\newcommand{\AND}{\;\&\;}
\newcommand{\shave}{}

%%%%%%%%%%%%%%%%% Maxell
\def\phie{\ensuremath{\phi_E}}
\def\phib{\ensuremath{\phi_B}}

\def\dB{\ensuremath{\emph{d}\vec{B}}}
\def\dphie{\ensuremath{\emph{d}\phie}}
\def\dphib{\ensuremath{\emph{d}\phib}}
\def\cint#1{\ensuremath{\displaystyle\underset{\substack{\text{\tiny{closed}}\\\text{\tiny{surface}}}}{\oint} \mspace{-0.1 mu} #1}}
\def\k{\ensuremath{\displaystyle \frac{1}{4\pi \epsilon _0}}}
\def\muz{\ensuremath{\mu_0}} %For some reason, \mu0 gave me an error - probably because \mu is a command
\providecommand{\Dotprod}[2]{#1 \bp #2}
\usepackage{graphicx,tikz,picins,phaistos}
\usepackage{subfig,dropping}
\usepackage{shadethm}
\usepackage{makeidx}
\usepackage{paralist}
\usepackage{mdwlist}
\usepackage{tkz-euclide}
\usepackage{tikz-cd}
\usepackage{tikz-3dplot}
\usepackage{multicol}
\usepackage{subfig,psfrag}
\usepackage{tikz}
\usetikzlibrary{matrix}
\usepackage{framed}
\usepackage{nextpage,listings}
\usetikzlibrary{matrix}
\usepackage[font=small,labelfont=bf,labelsep=space,format=plain]{caption}
\usepackage{longtable}
\usepackage[amsmath,thmmarks,standard,thref]{ntheorem}
\DeclareFixedFont{\bigithv}{T1}{phv}{b}{sl}{0.8cm}
% \usepackage[pdftitle={Vector Calculus}, pdfauthor={dmm}, bookmarksopen]{hyperref}
\def\msum{\operatornamewithlimits{\dsum^n\!{\ldots}\!\dsum^n}}
 \def\multsum{\mathop{\dsum^n\!{\ldots}\!\dsum^n}_{\substack{ h_1,\,\ldots,\,h_r\\
 k_1,\,\ldots,\,k_s}}}

\DeclareSymbolFont{largesymbols}{OMX}{yhex}{m}{n}

\DeclareMathAccent{\wideparen}{\mathord}{largesymbols}{"F3}
% \definecolor{Maroon}{cmyk}{0, 0.87, 0.68, 0.32}
\definecolor{RoyalBlue}{cmyk}{0.55,0.28,0.00,  0.42}
\definecolor{Maroon}{cmyk}{0.55,0.28,0.00,  0.42}

\makeatletter
\renewcommand\section{\@startsection {section}{1}{\z@}%
                                   {-3.5ex \@plus -1ex \@minus -.2ex}%
                                   {2.3ex \@plus.2ex}%
                                   {\bigithv}}
\makeatother
\Newassociation{answer}{Answer}{multicaca}
\usepackage[margin=2.2cm]{geometry}
\widowpenalty=1000 \clubpenalty=1000
%%%%%                   Postscript drawing packages
\usepackage{pst-math}
\usepackage{pst-eucl}
\usepackage{pst-grad}              % PSTricks package for gradient filling
\usepackage{pst-eps,pst-func,pstricks-add}
\usepackage{pst-solides3d}
\usepackage{pst-3dplot}
\usepackage{wrapfig}
\usepackage{graphicx}
\definecolor{mybluei}{RGB}{0,173,239}
\definecolor{myblueii}{RGB}{63,200,244}
\definecolor{myblueiii}{RGB}{199,234,253}

%%%%%                    THEOREM-LIKE ENVIRONMENTS
\newcommand{\proofsymbol}{\Pisymbol{pzd}{113}}
\renewcommand{\qedsymbol}{$\blacksquare$}
\renewenvironment{proof}{{\vspace{0.5cm}\noindent \bfseries Proof:}}{\qedsymbol \vspace{0.5cm}}
\theorembodyfont{\small}
\newtheorem{pro}{Problem}[section]
\theorempreskipamount .5cm \theorempostskipamount .5cm
\theoremstyle{changeb} \theoremheaderfont{\color{Maroon}\sffamily\bfseries}
\theorembodyfont{\normalfont}
\newtheorem{thm}{Theorem}[chapter]
\newtheorem{theoremeT}[thm]{Theorem}
\newtheorem{defn}{Definition}
\newtheorem{prop}[thm]{Proposition}
\newtheorem{cor}[thm]{Corollary}
\newtheorem{df}[thm]{Definition}
\newtheorem{lem}[thm]{Lemma}
\newtheorem{axi}[thm]{Axiom}
\newtheorem{nota}[thm]{Notation}
\newtheorem{exa}[thm]{Example}
\renewtheorem{theorem}[thm]{Theorem}
\renewtheorem{lemma}[thm]{Lemma}
\renewtheorem{definition}[thm]{Definition}
\renewtheorem{corollary}[thm]{Corollary}
\renewtheorem{proposition}[thm]{Proposition}
\renewtheorem{remark}[thm]{Remark}
\newenvironment{solu}[0]{{\noindent \textbf{ Solution:}  \ }$\blacktriangleright$  }{$\blacktriangleleft$}

\newenvironment{rem}[0]{\begin{quote}{ }}{\end{quote}}

\makeatletter
\def\La{%
  L\kern-.36em{%
  \setbox0\hbox{T}%
  \vbox to\ht0{%
    \hbox{$\m@th$%
      \csname S@\f@size\endcsname
      \fontsize\sf@size\z@
      \math@fontsfalse\selectfont A}%
      \vss%
    }%
  }%
} \makeatother


%%%%%%%

\setcounter{tocdepth}{1}

%%%%%%%%%%%%%%%%%%%%Two Column Table of Contents
\makeatletter
\newcommand{\twocoltoc}{%
  %\begin{multicols}{2}
    \@starttoc{toc}%
  %\end{multicols}
  }
\makeatother
%%%%%%%%%%%%
\makeindex
%%%%%                Non-standard commands and symbols
\newcommand{\BBZ}{\mathbb{Z}}
\newcommand{\reals}{\mathbb{R}}
\newcommand{\BBN}{\mathbb{N}}
\newcommand{\BBC}{\mathbb{C}}
\newcommand{\BBQ}{\mathbb{Q}}
\def\floor#1{\llfloor #1 \rrfloor}
\def\ang#1{\angle\left(#1\right)}
\def\binom#1#2{{#1\choose#2}}
\def\T#1#2{\mathscr{T}_{#1}{#2}}
\def\fun#1#2#3#4#5{\everymath{\displaystyle}{{#1} : \vspace*{1cm}
\begin{array}{lll}{#4} & \rightarrow &
{#5}\\
{#2} &  \mapsto & {#3} \\
\end{array}}}
\def\im#1{{\mathbf{Im}}\left(#1\right)}
\def\arc#1{\wideparen{#1}}
\def\dom#1{{\mathbf{Dom}}\left(#1\right)}
\def\supp#1{{\mathbf{supp}}\left(#1\right)}
\def\target#1{{\mathbf{Target}}\left(#1\right)}
\def\distance#1#2{{\mathbf d}{\langle {#1}, {#2}\rangle}}
\newcommand{\idefun}{{\mathbf Id\ }}
\def\tends#1#2{{#1 \rightarrow #2}}
\def\diff#1#2{\dfrac{\mathrm{d}}{\mathrm{d} #2}\ #1}
\def\dist{\mathrm d}
\def\bipoint#1#2{[\point{#1}, \point{#2}]}
\newcommand{\stackunder}[2]{\underset{\scriptstyle #1}{#2}}
\def\gl#1#2{{\mathbf GL}_{#1}(#2)}
\def\mat#1#2{{\mathbf M}_{#1}(#2)}
\def\norm#1{{\left\|#1\right\|}}
\def\absval#1{{\left|#1\right|}}
\newlength{\mylen}
\setlength{\mylen}{\dimexpr0.5\ht1-0.5\ht2}
\newcommand{\bp}{\raisebox{0.25ex}{\tiny$\bullet$}}
\def\dotprod#1#2{\vector{#1} \bp \vector{#2}}
\newcommand{\cross}{\boldsymbol\times}
\def\line#1{\overleftrightarrow{\point{#1}}}
\def\dd#1{\overline{#1}}
\def\seg#1{\left[ #1\right]}
\def\crossprod#1#2{\vector{#1}\cross\vector{#2}}
\def\anglebetween#1#2{\widehat{(\vector{#1}, \vector{#2})}}
\def\conv#1{{\mathrm conv}(#1)}
\def\tr#1{{\mathrm tr}\left(#1\right)}
\DeclareMathOperator{\adj}{adj}
\def\proj#1#2{{\mathrm proj} _{\point{#2}} ^{\point{#1}} }
\def\vol#1{{\mathrm volume}(#1)}
\def\ball#1#2{B_{#2}({\mathbf #1})}
\def\point#1{{\mathbf #1}}
\def\vector#1{\mathbf{ #1}}
\def\funvect#1{\mathbf{ #1}}
\def\deriv#1#2{{\mathrm{D}}_{\point{#1}}(#2)}
\def\derivb#1{{\mathrm{D}}(#1)}
\def\derivc{{\mathrm{D}}}
\def\D{{\mathrm{D}}}
\def\partialderiv#1#2{\partial_#1  #2}
\def\hderiv#1#2#3{\dfrac{\partial ^2 #3}{\partial x_{#1} \partial
x_{#2}}}
\def\div#1#2{{\mathrm div} #1{\left(\point{#2}\right)} }
\def\curl#1#2{{\mathrm curl} #1{\left(\point{#2}\right)} }
\def\distance#1#2{{\mathrm d}{\left(\vector{#1},\vector{#2}\right)} }
\def\rn#1#2#3{\langle \reals^#1,#2, #3\rangle}
\usepackage{xparse}
\NewDocumentCommand\Span{g}{%
    \ensuremath{ {\mathrm Span} \IfNoValueTF{#1}{}{\left(#1\right)} }%
}
\NewDocumentCommand\rank{g}{%
    \ensuremath{ {\text{ rank}} \IfNoValueTF{#1}{}{\left(#1\right)} }%
}

\NewDocumentCommand\sgn{g}{%
    \ensuremath{ {\mathrm sgn} \IfNoValueTF{#1}{}{\left(#1\right)} }%
}

\NewDocumentCommand\area{g}{%
    \ensuremath{ {\text{ area}} \IfNoValueTF{#1}{}{\left(#1\right)} }%
}
\newcommand{\TM}{\Pisymbol{psy}{228}\ }
\newcommand{\Surd}[1]{\left\delimiter"4270370 #1\right.}
\makeatletter
\def\clap#1{\hb@xt@\z@{\hss#1\hss}}
\makeatother
\def\partialda#1#2{\dfrac{\partial #2}{\partial {#1}}}
\def\partiald#1#2#3{\dfrac{\partial #2}{\partial x_{#1}}\left({\point{#3}}\right)}
\def\partialds#1#2#3{\partial_{#1} #2 \left({\point{#3}}\right)}
\def\partialdd#1#2#3#4{{\mathscr  D}_{#1}{\mathscr  D}_{#2}#3\left({\point{#4}}\right)}
\def\smallo#1#2{\begin{array}{l}{\mathbf o} \\ #1  \end{array}\left(#2\right)}
\def\smallosans#1{{\mathbf o}\left(#1\right)}
\def\jacobi#1#2{{\mathscr  J}_{#1}#2}
\def\hessian#1#2{{\mathscr  H}_{#1}#2}
% %%%%%%%%%%%%%%%%%INTERVALS
% %%%%%%%% lo= left open, rc = right closed, etc.
% \def\lcrc#1#2{\left[#1 \ ; #2 \right]}
% \def\loro#1#2{ \left]#1 \ ; #2 \right[}
% \def\lcro#1#2{\left[#1 \ ; #2 \right. \left[ \right.}
% \def\lorc#1#2{\left. \right[#1 \ ; #2 \left.\right]}

\def\es#1{\ensuremath{\textcolor{Maroon}{\ulcorner } #1  \textcolor{Maroon}{\lrcorner}}}
 \defineTColor{tRot}{red}
 \defineTColor{tCyanl}{cyan!60}
\defineTColor{tCyan}{cyan}
\defineTColor{tGelb}{yellow!60}

\newcommand{\grstep}[2][\relax]{%
   \ensuremath{\mathrel{
       \mathop{\longrightarrow}\limits^{#2\mathstrut}_{
                                     \begin{subarray}{l} #1 \end{subarray}}}}}
\newcommand{\colvec}[1]{\begin{bmatrix} #1 \end{bmatrix}}
\newcommand{\rowvec}[1]{\begin{bmatrix} #1 \end{bmatrix}}
\newcommand{\colpoint}[1]{\begin{bmatrix} #1 \end{bmatrix}}
\newcommand{\rowpoint}[1]{\begin{bmatrix} #1 \end{bmatrix}^\top}
\newcommand{\coord}[1]{\left( #1 \right)}

\newcommand{\dint}{\displaystyle\int}
% \renewcommand{\int}{\displaystyle\int\limits}
\newcommand{\doint}{\displaystyle\oint}
\newcommand{\doiint}{\displaystyle\oiint}

\newcommand{\diint}{\displaystyle\iint}
\newcommand{\diiint}{\displaystyle\iiint}

\RequirePackage{array}\RequirePackage{dcolumn}
\newenvironment{aligncolondecimal}[2][.1111em]{%
\setlength{\arraycolsep}{#1}
\newcolumntype{.}{D{.}{.}{#2}}\begin{array}{.}}{%
\end{array}}
\title{Vector Calculus}

% \usepackage{sectsty}
% \chapterfont{\color{Maroon}} 
% \sectionfont{\color{Maroon}} 
% \subsectionfont{\color{Maroon}} 

\renewcommand{\arraystretch}{1.3}
\renewcommand{\d}{\dif}
\def\bluesum{\operatorname*{\blue{\sum}}}

\tikzset{
addarrow/.style={postaction={decorate},
        decoration={markings,mark=at position #1 with {\arrow{>}}}}
}

\newcommand{\christoffel}[3]{\ensuremath{\Gamma_{#1#2}^{#3}}}
\newcommand{\dvb}{\grad \bp}
\usetikzlibrary{calc,intersections}
 \usetikzlibrary{matrix}

\usetikzlibrary{arrows}
 \usetikzlibrary{decorations.markings}
\usetikzlibrary{decorations.pathmorphing}
 \usetikzlibrary{positioning}
 \usetikzlibrary{cd}
\addbibresource{bibliografia.bib}
\tikzset{circ/.style = {fill, circle, inner sep = 0, minimum size = 3}}
 \newcommand{\todoin}[2][]{\todo[inline,backgroundcolor=RoyalBlue!20,#1,caption=#2]{\textbf{Todo:\\}#2}}   
\def\lcm#1{{\mathrm lcm}\left(#1\right)}

  \usepackage[amsmath,thmmarks,standard,thref]{ntheorem}
\RequirePackage[framemethod=default]{mdframed} % Required for creating the theorem, definition, exercise and corollary boxes


\definecolor{azul}{RGB}{30,144,255}
\definecolor{azulescuro}{RGB}{0.252, 0.419,0.584}
\definecolor{verde}{RGB}{127, 137, 97}
\definecolor{vermelho}{cmyk}{0, 0.87, 0.68, 0.32}
\definecolor{laranja}{RGB}{243,102,25}
\definecolor{blue}{rgb}{0.252, 0.419,0.584}
\definecolor{red}{cmyk}{0, 0.87, 0.68, 0.32}
\definecolor{amarelinho}{RGB}{255,249,215}
\definecolor{azulzinho}{RGB}{238,247,250}
\definecolor{mybluei}{RGB}{0,173,239}
\definecolor{myblueii}{RGB}{63,200,244}
\definecolor{myblueiii}{RGB}{199,234,253}

% Definition box
\newmdenv[skipabove=13pt,
backgroundcolor=mybluei!4,
skipbelow=20pt,
rightline=true,
leftline=true,
topline=true,
bottomline=true,
linecolor=mybluei!20,
innerleftmargin=5pt,
innerrightmargin=5pt,
innertopmargin=10pt,
leftmargin=0.1cm,
rightmargin=0.1cm,
linewidth=0.5pt,
innerbottommargin=7pt,nobreak=true]{dBox}

% \newtheoremstyle{azulnumbox} % Theorem style name
% {0pt}% Space above
% {0pt}% Space below
% {\normalfont}% Body font
% {}% Indent amount
% {\small\bfseries\sffamily \color{azul} }% Theorem head font
% {\;}% Punctuation after theorem head
% {0.25em}% Space after theorem head
% {\small\sffamily\thmname{#1}\nobreakspace\thmnumber{\@ifnotempty{#1}{}\@upn{#2}}% Theorem text (e.g. Theorem 2.1)
% \thmnote{\nobreakspace\the\thm@notefont\sffamily\bfseries---\nobreakspace#3.}}% Optional theorem note
% 
% 
% 
% \theoremstyle{azulnumbox}

\newmdenv[skipabove=14pt,
skipbelow=20pt,
backgroundcolor=verde!10,
linecolor=verde!36,
innerleftmargin=5pt,
innerrightmargin=5pt,
innertopmargin=10pt,
leftmargin=0.1cm,
rightmargin=0.1cm,
innerbottommargin=10pt,nobreak=true]{tBox}
\newtheorem{definitionT}[thm]{Definition}


\newmdenv[skipabove=14pt,
skipbelow=20pt,
backgroundcolor=verde!5,
linecolor=verde!26,
innerleftmargin=5pt,
innerrightmargin=5pt,
innertopmargin=10pt,
leftmargin=0.1cm,
rightmargin=0.1cm,
innerbottommargin=10pt,nobreak=true]{pBox}
\newtheorem{propT}[thm]{Proposition}


\mdfsetup{skipabove=10pt,skipbelow=15pt}
\renewenvironment{df}{\begin{dBox}\begin{definitionT}}{\end{definitionT}\end{dBox}\vspace{0.5cm}}
\renewenvironment{definition}{\begin{dBox}\begin{definitionT}}{\end{definitionT}\end{dBox}\vspace{0.5cm}}




\renewenvironment{theorem}{\begin{tBox}\begin{theoremeT}}{\end{theoremeT}\end{tBox}\vspace{0.5cm}}
\renewenvironment{thm}{\begin{tBox}\begin{theoremeT}}{\end{theoremeT}\end{tBox}\vspace{0.5cm}}

\renewenvironment{proposition}{\begin{pBox}\begin{propT}}{\end{propT}\end{pBox}\vspace{0.5cm}}
\renewenvironment{prop}{\begin{pBox}\begin{propT}}{\end{propT}\end{pBox}\vspace{0.5cm}}


\newcommand{\superficie}{S}
\newcommand{\caminho}{\gamma}
\renewcommand{\S}{\mathrm{S}}

\newcommand{\dA}{\mathrm{dA}}
\newcommand{\dV}{\mathrm{dV}}

% \renewcommand{\ell}{\vector{r}}

\def\versor#1{\mathbf{\hat  #1}}

\tikzset{declare function={f(\x)=sin(\x*100)/10;},
  non-linear cs/.cd,
    x/.store in=\nlx,y/.store in=\nly,z/.store in=\nlz,
    x=0,y=0,z=0}
\tikzdeclarecoordinatesystem{non-linear}{%
  \tikzset{non-linear cs/.cd,#1}%
  \pgfpointxyz{(\nlx)-f(\nly)*3+f(\nlz)}%
    {-f(\nlx)*2+(\nly)-f(\nlz)}{-f(\nlx)-f(\nly)+(\nlz)}}
