\chapter{Determinants}

\section{Permutations}
\begin{df}
Let $S$ be a finite set with $n \geq 1$ elements. A \negrito{permutation} is a bijective function $\tau : S \rightarrow S$. It
is easy to see that there are $n!$ permutations from $S$ onto
itself. \index{permutation}
\end{df}
Since we are mostly concerned with the {\em action} that $\tau$
exerts on $S$ rather than with the particular names of the
elements of $S$, we will take $S$ to be the set $S = \{1, 2, 3,
\ldots , n \}$. We indicate a permutation $\tau$ by means of the
following convenient diagram
$$\tau = \begin{bmatrix} 1 & 2 & \cdots & n \cr \tau (1) & \tau (2) & \cdots & \tau (n) \end{bmatrix}.$$
\begin{df}
The notation $S_n$ will denote the set of all permutations on
$\{1, 2, 3, \ldots , n \}$. Under this notation, the  composition
of two permutations $(\tau, \sigma)\in S_n ^2$ is
$$ \begin{array}{lll}\tau\circ \sigma & = & \begin{bmatrix} 1 & 2 & \cdots & n \cr \tau (1) & \tau (2) & \cdots & \tau (n) \end{bmatrix}
\circ\begin{bmatrix} 1 & 2 & \cdots & n \cr \sigma (1) & \sigma
(2) & \cdots & \sigma (n) \end{bmatrix} \\ &  = &
\begin{bmatrix} 1 & 2 & \cdots & n \cr (\tau \circ \sigma)  (1) & (\tau \circ \sigma)  (2) & \cdots &
(\tau \circ \sigma) (n) \end{bmatrix}\end{array}.  $$ The $k$-fold
composition of $\tau$ is
$$ \underbrace{\tau \circ \cdots \circ \tau}_{k\ {\textrm  compositions}} = \tau^k.       $$

\end{df}

\begin{rem}
We usually do away with the $\circ$ and write $\tau\circ\sigma$
simply as $\tau\sigma$. This ``product of permutations'' is thus
simply function composition.
\end{rem}
 Given a permutation
$\tau: S \rightarrow S$, since $\tau$ is bijective,
$$\tau ^{-1}: S \rightarrow S$$ exists and
is also a permutation. In fact if
$$\tau = \begin{bmatrix} 1 & 2 & \cdots & n \\ \tau (1) & \tau (2) & \cdots & \tau (n) \end{bmatrix},$$
then
$$\tau^{-1} = \begin{bmatrix}\tau (1) & \tau (2) & \cdots & \tau (n) \\ 1 & 2 & \cdots & n \\  \end{bmatrix}.$$
\vspace{2cm}
\begin{figure}[htb]
$$\psset{unit=2pc}\psline[linewidth=2pt](-1, 0)(1, 0)(0, 1.4142135623730950488016887242097)(-1,0)
\psline[linestyle=dotted](0,1.4142135623730950488016887242097)(0,0)
\uput[u](0, 1.4142135623730950488016887242097){1}
\psline[linestyle=dotted](-1,0)(.5,
0.70710678118654752440084436210485) \uput[l](-1, 0){2}
\psline[linestyle=dotted](1,0)(-.5,
0.70710678118654752440084436210485) \uput[r](1,0){3}
$$\vspace{1cm}\caption{$S_3$ are rotations and reflexions.} \label{fig:S_3}
\end{figure}

\begin{exa}
The set $S_3$ has  $3! = 6$ elements, which are given below.
\begin{enumerate}
\item $\idefun: \{1, 2, 3\} \rightarrow \{1, 2, 3\}$ with
$$\idefun  = \begin{bmatrix} 1 & 2 & 3 \\  1 & 2 & 3 \end{bmatrix}.$$
\item $\tau_1: \{1, 2, 3\} \rightarrow \{1, 2, 3\} $ with
$$\tau_1  = \begin{bmatrix}1 & 2 & 3 \\  1 & 3 & 2 \end{bmatrix}.$$

\item $\tau_2: \{1, 2, 3\}  \rightarrow \{1, 2, 3\} $ with
$$\tau_2  = \begin{bmatrix} 1 & 2 & 3 \\ 3 & 2 & 1 \end{bmatrix}.$$
\item $\tau_3: \{1, 2, 3\}  \rightarrow \{1, 2, 3\} $ with
$$\tau_3  = \begin{bmatrix} 1 & 2 & 3 \\ 2 & 1 & 3 \end{bmatrix}.$$

\item $\sigma_1: \{1, 2, 3\}  \rightarrow \{1, 2, 3\} $ with
$$\sigma_1  = \begin{bmatrix} 1 & 2 & 3 \\ 2 & 3 & 1 \end{bmatrix}.$$

\item $\sigma_2: \{1, 2, 3\} \rightarrow \{1, 2, 3\} $ with
$$\sigma_2  = \begin{bmatrix} 1 & 2 & 3 \\ 3 & 1 & 2 \end{bmatrix}.$$

\end{enumerate}
\label{ex:S_3}\end{exa}

\begin{exa}
The compositions $\tau_1 \circ \sigma_1$ and $\sigma_1\circ
\tau_1$ can be found as follows.
$$ \tau_1 \circ \sigma_1  = \begin{bmatrix} 1 & 2 & 3 \\  1 & 3 & 2 \end{bmatrix}\circ
\begin{bmatrix} 1 & 2 & 3 \\ 2 & 3 & 1
\end{bmatrix} = \begin{bmatrix}1 & 2 & 3 \\ 3 & 2 & 1 \end{bmatrix} = \tau_2 .$$
(We read from right to left $1 \rightarrow 2 \rightarrow 3$ (``1
goes to 2, 2 goes to 3, so 1 goes to 3''), etc. Similarly
$$ \sigma_1 \circ \tau_1  = \begin{bmatrix} 1 & 2 & 3 \\ 2 & 3 & 1
\end{bmatrix}\circ\begin{bmatrix}1 & 2 & 3 \\  1 & 3 & 2 \end{bmatrix}
 = \begin{bmatrix} 1 & 2 & 3 \\ 2 & 1 & 3 \end{bmatrix} =
 \tau_3.$$Observe in particular that $\sigma_1 \circ \tau_1  \neq \tau_1 \circ \sigma_1 .$
 Finding all the other products we deduce the following
 ``multiplication table'' (where the ``multiplication'' operation
 is really composition of functions).

$$\begin{array}{|c||c|c|c|c|c|c|}
\hline \circ & \idefun & \tau_1 & \tau_2 & \tau_3  & \sigma_1 & \sigma_2\\
\hline \hline  \idefun & \idefun & \tau_1  & \tau_2  & \tau_3 & \sigma_1 & \sigma_2   \\
\hline  \tau_1 & \tau_1 &   \idefun & \sigma_1 &\sigma_2  & \tau_2 & \tau_3   \\
\hline \tau_2 & \tau_2 & \sigma_2  & \idefun & \sigma_1 & \tau_3&\tau_1 \\
\hline \tau_3 & \tau_3 & \sigma_1& \sigma_2 & \idefun &\tau_1 & \tau_2 \\
\hline \sigma_2 & \sigma_2 & \tau_2& \tau_3& \tau_1& \idefun & \sigma_1 \\
\hline \sigma_1 & \sigma_1 &  \tau_3 & \tau_1 & \tau_2& \sigma_2 &\idefun \\
\hline
\end{array}$$
\label{ex:tau}\end{exa}

The permutations in example \ref{ex:S_3} can be conveniently
interpreted as follows. Consider an equilateral triangle with
vertices labelled $1$, $2$ and $3$, as in figure \ref{fig:S_3}.
Each $\tau_a$ is a reflexion (``flipping'') about the line joining
the vertex $a$ with the midpoint of the side opposite $a$. For
example $\tau_1$ fixes $1$ and flips $2$ and $3$. Observe that two
successive flips return the vertices to their original position
and so $(\forall a\in\{1, 2, 3\})(\tau_a ^2 = \idefun)$.
Similarly, $\sigma_1$ is a rotation of the vertices by an angle of
$120^\circ$. Three successive rotations restore the vertices to
their original position and so $\sigma_1 ^3 = \idefun$.


\begin{exa}
To find $\tau_1^{-1}$ take the representation of $\tau_1$ and
exchange the rows:
$$ \tau_1^{-1}  = \begin{bmatrix}1 & 3 & 2 \\  1 & 2 & 3 \\  \end{bmatrix}.$$
This is more naturally written as
$$ \tau_1^{-1}  = \begin{bmatrix}1 & 2 & 3 \\  1 & 3 & 2 \\  \end{bmatrix}.$$
Observe that $\tau_1^{-1} = \tau _1$.
\end{exa}
\begin{exa}
To find $\sigma_1^{-1}$ take the representation of $\sigma_1$ and
exchange the rows:
$$ \sigma_1^{-1}  = \begin{bmatrix}2 & 3 & 1 \\  1 & 2 & 3 \\  \end{bmatrix}.$$
This is more naturally written as
$$ \sigma_1^{-1}  = \begin{bmatrix}1 & 2 & 3 \\  3 & 1 & 2 \\  \end{bmatrix}.$$
Observe that $\sigma_1^{-1} = \sigma_2$.
\end{exa}




\section{Cycle Notation} We now present a
shorthand notation for permutations by introducing the idea of a
{\em cycle}. Consider in $S_9$ the permutation
$$\tau = \begin{bmatrix} 1 & 2 & 3 & 4 & 5 & 6 & 7 & 8 & 9 \\
2 & 1 & 3 & 6 & 9 & 7 & 8 & 4 & 5 \end{bmatrix}.$$ We start with
1. Since 1 goes to 2 and 2 goes back to 1, we write $(12)$. Now we
continue with 3. Since 3 goes to 3, we write $(3)$. We continue
with 4. As 4 goes 6, 6 goes to 7, 7 goes 8, and 8 goes back to 4,
we write $(4678)$. We consider now $5$ which goes to 9 and 9 goes
back to 5, so we write $(59)$. We have written $\tau$ as a product
of disjoint cycles
$$\tau = (12)(3)(4678)(59).$$This prompts the following
definition.
\begin{df}
Let $l\geq 1$ and let $i_1,\ldots,i_l\in\{1,2,\ldots n\}$ be
distinct. We write  $(i_1\ i_2\ \ldots\ i_l)$ for the element
$\sigma\in S_n$ such that $\sigma(i_r)=i_{r+1}, \ \ 1\leq r<l$,
$\sigma(i_l)=i_1$ and  $\sigma(i)=i$ for
$i\not\in\{i_1,\ldots,i_l\}$. We say that  $(i_1\ i_2\ \ldots\
i_l)$ is a \negrito{cycle of length $l$}. The \negrito{order} of a cycle
is its length. Observe that if $\tau$ has order $l$ then $\tau^l =
\idefun$.\end{df}
\begin{rem}
Observe that $(i_2\ \ldots\ i_l\ i_1)=(i_1\ \ldots\ i_l)$ etc.,
and that $(1)=(2)=\cdots=(n)= \idefun$. In fact, we have
$$(i_1\ \ldots\ i_l)=(j_1\ \ldots\ j_m)$$ if and only if (1) $l=m$
and if  (2) $l>1$: $\exists a$ such that $\forall k$: $i_k =
j_{k+a \mod l}$. Two cycles $(i_1,\ldots,i_l)$ and
$(j_1,\ldots,j_m)$ are disjoint if $\{i_1, \ldots ,
i_l\}\cap\{j_1,  \ldots , j_m\}=\varnothing$. Disjoint cycles
commute and if $\tau = \sigma_1\sigma_2 \cdots \sigma_t$ is the
product of disjoint cycles of length $l_1, l_2, \ldots , l_t$
respectively, then $\tau$ has order
$$ \lcm{l_1, l_2, \ldots , l_t}.$$
\end{rem}
\begin{exa}
A cycle decomposition for $\alpha\in S_9,$
$$\alpha = \begin{bmatrix} 1 & 2 & 3 & 4 & 5 & 6 & 7 & 8 & 9 \\
1 & 8 & 7 & 6 & 2 & 3 & 4 & 5 & 9 \end{bmatrix}$$ is
$$(285)(3746).$$ The order of $\alpha$ is $\lcm{3, 4} = 12$.
\label{ex:alpha}\end{exa}
\begin{exa}
The cycle decomposition $\beta = (123)(567)$ in $S_9$ arises from
the permutation
$$\beta = \begin{bmatrix}1 & 2 & 3 & 4 & 5 & 6 & 7 & 8 & 9 \\
2 & 3 & 1 & 4 & 6 & 7 & 5 & 8 & 9 \\ \end{bmatrix}.$$ Its order is
$\lcm{3,3} = 3$. \label{ex:beta}\end{exa}
\begin{exa}

Find a shuffle of a deck of $13$ cards that requires $42$ repeats
to return the cards to their original order.
\end{exa}
\begin{solu}Here is one (of many possible ones). Observe that $7 + 6
= 13$ and $7\times 6 = 42$. We take the permutation
$$(1\ 2\ 3\ 4\ 5\ 6\ 7)(8\ 9\ 10\ 11\ 12\ 13)$$which has order 42.
This corresponds to the following shuffle: For $$i\in \{1, 2, 3,
4, 5, 6, 8, 9, 10, 11, 12\},$$ take the $i$th card to the ($i +
1$)th place, take the $7$th card to the first position and the
13th card to the $8$th position. Query: Of all possible shuffles
of 13 cards, which one takes the longest to restitute the cards to
their original position?
\end{solu}
\begin{exa}
Let a shuffle of a deck of 10 cards be made as follows: The top
card is put at the bottom, the deck is cut in half, the bottom
half is placed on top of the top half, and then the resulting
bottom card is put on top. How many times must this shuffle be
repeated to get the cards in the initial order? Explain.
\end{exa}
\begin{solu}Putting the top card at the bottom corresponds to
$$\begin{bmatrix} 1 & 2 & 3 & 4 & 5 & 6 & 7  & 8 & 9 & 10\cr
2 & 3 & 4 & 5 & 6 & 7  & 8 & 9 & 10 & 1 \cr
\end{bmatrix}.
$$ Cutting this new arrangement in half and putting the lower half on top corresponds to
$$\begin{bmatrix} 1 & 2 & 3 & 4 & 5 & 6 & 7  & 8 & 9 & 10\cr
 7  & 8 & 9 & 10 & 1 & 2 & 3 & 4 & 5 & 6 \cr
\end{bmatrix}.
$$
Putting the bottom card of this new arrangement on top corresponds
to
$$\begin{bmatrix} 1 & 2 & 3 & 4 & 5 & 6 & 7  & 8 & 9 & 10\cr
  6 & 7  & 8 & 9 & 10 & 1 & 2 & 3 & 4 & 5  \cr
\end{bmatrix} = (1\ 6)(2\ 7)(3\ 8)(4\ 9)(5\ 10).
$$The order of this permutation is ${\textrm  lcm} (2, 2, 2, 2, 2) = 2$, so
in 2 shuffles the cards are restored to their original position.
\end{solu}
The above examples illustrate the general case, given in the
following theorem.
\begin{thm}
Every permutation in $S_n$ can be written as a product of disjoint
cycles.\label{thm:permu_is_prod_cycles}
\end{thm}
\begin{proof}
Let $\tau \in S_n, a_1\in\{1, 2, \ldots, n\}$. Put $\tau ^k (a_1)
= a_{k + 1}, k \geq 0$. Let $a_1, a_2, \ldots , a_s$ be the
longest chain with no repeats. Then we have $\tau (a_s) = a_1$. If
the  $\{a_1, a_2, \ldots , a_s\}$ exhaust $\{1, 2, \ldots, n\}$
then we have $\tau = (a_1\ a_2\ \ldots\ a_s)$. If not, there exist
$b_1\in \{1, 2, \ldots, n\} \setminus\{a_1, a_2, \ldots , a_s\}$.
Again, we find the longest chain of distinct $b_1, b_2, \ldots,
b_t$ such that $\tau (b_k) = b_{k + 1}, k = 1, \ldots, t -1$ and
$\tau (b_t) = b_1.$ If  the $\{a_1, a_2, \ldots , a_s, b_1, b_2,
\ldots, b_t\}$ exhaust all the $\{1, 2, \ldots, n\}$ we have $\tau
= (a_1\ a_2\ \ldots\ a_s)( b_1\ b_2\ \ldots\ b_t)$. If not we
continue the process and find $$\tau = (a_1\ a_2\ \ldots\ a_s)(
b_1\ b_2\ \ldots\ b_t)(c_1\ldots )\ldots . $$This process stops
because we have only $n$ elements.
\end{proof}
\begin{df}
A \negrito{transposition} is a cycle of length $2$.\footnote{A cycle
of length $2$ should more appropriately be called a \negrito{bicycle}.}
\end{df}


\begin{exa}
The cycle $(23468)$ can be written as a product of transpositions
as follows
$$(23468) = (28)(26)(24)(23).$$ Notice that this decomposition as
the product of transpositions is not unique. Another decomposition
is $$(23468) = (23)(34)(46)(68).$$
\end{exa}
\begin{lemma}
Every permutation is the product of transpositions.
\label{lem:permu_is_prod_bicycles}\end{lemma}
\begin{proof}
It is enough to observe that
$$ (a_1\ a_2\ \ldots\ a_s) = (a_1\ a_s)(a_1\ a_{s - 1}) \cdots (a_1\ a_2)
$$and appeal to Theorem \ref{thm:permu_is_prod_cycles}.
\end{proof}
Let  $\sigma\in S_n$ and let $(i, j)\in \{1, 2, \ldots, n\}^2, \ \
i\neq j$. Since $\sigma$ is a permutation, $\exists (a, b)\in \{1,
2, \ldots, n\}^2, \ \  a\neq b$, such that $\sigma (j) - \sigma
(i) = b - a$. This means that $$ \left|\prod_{1\leq i<j\leq n}
\frac{\sigma(i)-\sigma(j)}{i-j}\right| = 1 . $$
\begin{df} Let  $\sigma\in S_n$. We define the \negrito{sign} $\sgn\sigma$ of $\sigma$ as
$$\sgn{\sigma} = \prod_{1\leq i<j\leq n}
\frac{\sigma(i)-\sigma(j)}{i-j} = (-1)^{\sigma}.$$ If
$\sgn{\sigma} = 1$, then we say that $\sigma$ is an \negrito{even permutation}, and if $\sgn{\sigma} = -1$ we say that $\sigma$ is
an \negrito{odd permutation}.
\end{df}
\begin{rem}
Notice that in fact $$\sgn{\sigma} =(-1)^{{\textbf  I}(\sigma)},$$where
${\textbf  I}(\sigma)=\#\{(i,j)\,|\,1\leq i<j\leq n\;{\textrm  and
}\;\sigma(i)>\sigma(j)\}$, i.e., ${\textbf  I}(\sigma)$ is the number
of inversions that $\sigma$ effects to the identity permutation
$\idefun$.\end{rem}
\begin{exa}
The transposition $(1\ 2)$ has one inversion.
\end{exa}
\begin{lemma}
 For any transposition $(k\ l)$ we have $\sgn{(k\ l)} = -1$.
\end{lemma}
\begin{proof}
Let $\tau$ be transposition that exchanges $k$ and $l$, and assume
that $k<l$:
$$
 \tau=
\begin{bmatrix}
1 & \dots & k-1 & k &  \dots & l-1 & l & l+1 & \dots & n \\
1 & \dots & k-1 & l &  \dots & l-1 & k & l+1 & \dots & n
\end{bmatrix}
$$
Let us count the number of inversions of $\tau$:
\begin{itemize}
 \item The pairs $(i,j)$ with $i\in\{1, 2, \ldots, k - 1\}\cup\{l, l + 1, \ldots, n\}$ and $i<j$
 do not suffer an inversion;
 \item The pair $(k,j)$ with $k<j$ suffers an inversion if and only if
 $j\in \{k+1, k + 2, \ldots , l\}$, making  $l-k$
inversions;
 \item If $i\in\{k+1, k + 2, \ldots , l-1\}$ and $i<j$, $(i,j)$
 suffers an
inversion if and only if $j=l$, giving $l-1-k$ inversions.
\end{itemize}
This gives a total of ${\textbf  I}(\tau)=(l-k)+(l-1-k)=2(l-k-1)+1$
inversions when $k < l$. Since this number is odd, we have
$\sgn{\tau}=(-1)^{{\textbf  I}(\tau)}=-1$. In general we see that the
transposition $(k\ l)$ has $2|k - l| - 1$ inversions.
\end{proof}
\begin{thm} \label{thm:signum_is_a_homomorphism} Let $(\sigma, \tau)\in S_n
^2$. Then
$$ \sgn{\tau\sigma} = \sgn{\tau}\sgn{\sigma}. $$
\end{thm}
\begin{proof}
We have
$$\begin{array}{lll}
\sgn{\sigma\tau} & = &  \prod_{1\leq i<j\leq n}
\frac{(\sigma\tau)(i)-(\sigma\tau)(j)}{i-j} \\ & = &
\left(\prod_{1\leq i<j\leq n}
\frac{\sigma(\tau(i))-\sigma(\tau(j))}
{\tau(i)-\tau(j)}\right)\cdot \left(\prod_{1\leq i<j\leq n}
\frac{\tau(i)-\tau(j)}{i-j}\right). \end{array}
$$
The second factor on this last equality is clearly  $\sgn{\tau}$,
we must shew that the first factor is $\sgn{\sigma}$. Observe now
that for  $1\leq a<b\leq n$ we have
$$
\frac{\sigma(a)-\sigma(b)}{a-b}=\frac{\sigma(b)-\sigma(a)}{b-a}.
$$Since
$\sigma$ and $\tau$ are permutations, $\exists b \neq a, \ \tau
(i) = a, \tau (j) = b$ and so $\sigma\tau (i) = \sigma (a),
\sigma\tau (j) = b$. Thus
$$ \frac{\sigma(\tau(i))-\sigma(\tau(j))}
{\tau(i)-\tau(j)} = \frac{\sigma (a) - \sigma (b)}{a - b} $$ and
so
$$\prod_{1\leq
i<j\leq n} \frac{\sigma(\tau(i))-\sigma(\tau(j))}
{\tau(i)-\tau(j)} = \prod_{1\leq a<b\leq n}
\frac{\sigma(a)-\sigma(b)} {a - b} = \sgn{\sigma}.$$
\end{proof}
\begin{cor}
The identity permutation is even. If $\tau\in S_n$, then
$\sgn{\tau} = \sgn{\tau^{-1}}$.
\label{cor:idefun_is_even}\end{cor}
\begin{proof}
Since there are no inversions in $\idefun$, we have $\sgn{\idefun}
= (-1)^0 = 1$. Since $\tau\tau^{-1} = \idefun$, we must have $1 =
\sgn{\idefun} = \sgn{\tau\tau^{-1}} = \sgn{\tau}\sgn{\tau^{-1}} =
(-1)^\tau (-1)^{\tau^{-1}}$ by Theorem
\ref{thm:signum_is_a_homomorphism}. Since the values on the
righthand of this last equality are $\pm 1$, we must have
$\sgn{\tau}=\sgn{\tau^{-1}}$.
\end{proof}
\begin{lemma}
We have $\sgn{1\ 2\ \ldots\ l)}=(-1)^{l-1}$.
\label{lem:inv_in_1_through_l}\end{lemma}
\begin{proof}
Simply observe that the number of inversions of $(1\ 2\ \ldots\
l)$ is $l- 1$.
\end{proof}
\begin{lemma} \label{lem:signum_through_cycles}
Let $(\tau, (i_1\ \ldots\ i_l)\in S_n ^2$. Then
$$
\tau(i_1\ \ldots\ i_l)\tau^{-1} = (\tau(i_1)\ \ldots\
\tau(i_l)),$$ and if $\sigma\in S_n$ is a  cycle of length $l$
then
$$\sgn{\sigma}=(-1)^{l-1}$$.\\
\end{lemma}
\begin{proof}
For $1\leq k<l$ we have $(\tau(i_1\ \ldots\
i_l)\tau^{-1})(\tau(i_k)) = \tau((i_1\ \ldots\ i_l)(i_k)) =
\tau(i_{k+1})$. On a $(\tau(i_1\ \ldots\ i_l)\tau^{-1})(\tau(i_l))
= \tau((i_1\ \ldots\ i_l)(i_l)) = \tau(i_1)$. For
$i\not\in\{\tau(i_1)\ \ldots\ \tau(i_l)\}$ we have
$\tau^{-1}(i)\not\in\{i_1\ \ldots\ i_l\}$ whence $(i_1\ \ldots\
i_l)(\tau^{-1}(i))=\tau^{-1}(i)$ etc.

\bigskip

Furthermore, write $\sigma=(i_1\ \ldots\ i_l)$. Let $\tau\in S_n$
be such that  $\tau(k)=i_k$ for $1\leq k\leq l$. Then
$\sigma=\tau(1\ 2\ \ldots\ l)\tau^{-1}$ and so we must have
$\sgn{\sigma} = \sgn{\tau}\sgn{(1\ 2\ \ldots\ l)}\sgn{\tau^{-1}}$,
which equals  $\sgn{(1\ 2\ \ldots\ l)}$ by virtue of Theorem
\ref{thm:signum_is_a_homomorphism} and Corollary
\ref{cor:idefun_is_even}. The result now follows by appealing to
Lemma \ref{lem:inv_in_1_through_l}
\end{proof}
\begin{cor}\label{cor:signum_through_bikes}
Let $\sigma=\sigma_1\sigma_2\cdots\sigma_r$ be a product of
disjoint cycles, each of length  $l_1,\ldots,l_r$, respectively.
Then
$$
\sgn{\sigma} = (-1)^{\sum_{i=1}^r (l_i-1)}.
$$Hence, the product of two even permutations is even, the product
of two odd permutations is even, and the product of an even
permutation and an odd permutation is odd.
\end{cor}
\begin{proof}
This follows at once from Theorem
\ref{thm:signum_is_a_homomorphism} and Lemma
\ref{lem:signum_through_cycles}.
\end{proof}
\begin{exa}
The cycle $(4678)$ is an odd cycle; the cycle $(1)$ is an even
cycle; the cycle $(12345)$ is an even cycle.
\end{exa}
\begin{cor}\label{cor:permu_is_prod_of_bikes}
Every permutation can be decomposed as a product of
transpositions. This decomposition is not necessarily unique, but
its parity is unique.
\end{cor}
\begin{proof}
This follows from Theorem \ref{thm:permu_is_prod_cycles}, Lemma
\ref{lem:permu_is_prod_bicycles}, and Corollary
\ref{cor:signum_through_bikes}.
\end{proof}
\begin{exa}[The $15$ puzzle] Consider a grid with $16$ squares, as shewn in (\ref{eq:15_puzzle_1}), where $15$ squares are numbered $1$ through $15$ and the 16th slot is
empty.
\begin{equation}\label{eq:15_puzzle_1}
\begin{array}{|c|c|c|c|}
\hline 1 & 2 & 3 & 4 \\ \hline 5 & 6 & 7 & 8 \\ \hline 9 & 10 & 11
& 12 \\ \hline 13 & 14 & 15 & \\ \hline
\end{array}\end{equation}
In this grid we may successively exchange the empty slot with any
of its neighbours, as for example
\begin{equation}\label{eq:15_puzzle_2}
\begin{array}{|c|c|c|c|}
\hline 1 & 2 & 3 & 4 \\ \hline 5 & 6 & 7 & 8 \\ \hline 9 & 10 & 11
& 12 \\ \hline 13 & 14 &  & 15 \\ \hline
\end{array}.
\end{equation}We ask whether through a series of valid moves we may arrive at the following position.
\begin{equation}\label{eq:15_puzzle_3}
\begin{array}{|c|c|c|c|}
\hline 1 & 2 & 3 & 4 \\ \hline 5 & 6 & 7 & 8 \\ \hline 9 & 10 & 11
& 12 \\ \hline 13 & 15 & 14 &  \\ \hline
\end{array}
\end{equation}
\end{exa}
\begin{solu}Let us shew that this is impossible. Each time we move a
square to the empty position, we make transpositions on the set
$\{1,2,\ldots,16\}$. Thus at each move, the permutation is
multiplied by a transposition and hence it changes sign. Observe
that the permutation corresponding to the square in
(\ref{eq:15_puzzle_3}) is  $(14\ 15)$ (the positions 14th and 15th
are transposed) and hence it is an odd permutation. But we claim
that the empty slot can only return to its original position after
an even permutation. To see this paint the grid as a checkerboard:
\begin{equation}\label{eq:15_puzzle_4}
\begin{array}{|c|c|c|c|} \hline
B & R & B & R\\ \hline R & B & R & B \\ \hline B & R & B & R\\
\hline R & B & R & B \\ \hline
\end{array}
\end{equation}
We see that after each move, the empty square changes from black
to red, and thus after an odd number of moves the empty slot is on
a red square. Thus the empty slot cannot return to its original
position in an odd number of moves. This completes the proof.
\end{solu}
\section*{\psframebox{Homework}}

\begin{pro}
Decompose the permutation
$$\begin{bmatrix} 1 & 2 & 3 & 4 & 5 & 6 & 7 & 8 & 9 \cr
2 & 3 & 4 & 1 & 5 & 8 & 6 & 7 & 9 \end{bmatrix}$$as a product of
disjoint cycles and find its order. \begin{answer} This is clearly
$(1\ 2\ 3 \ 4)(6\ 8\ 7) $ of order ${\textrm  lcm} (4, 3) = 12$.
\end{answer}
\end{pro}

\section{Determinants} There are many ways of
developing the theory of determinants. We will choose a way that
will allow us to deduce the properties of determinants with ease,
but has the drawback of being computationally cumbersome. In the
next section we will shew that our way of defining determinants is
equivalent to a more computationally friendly one.
\bigskip

It may be pertinent here to quickly review some properties of
permutations. Recall that if $\sigma \in S_n$ is a cycle of length
$l$, then its signum $\sgn{\sigma} = \pm 1$ depending on  the
parity of $l - 1$. For example, in $S_7$, $$\sigma = (1 \ 3 \ 4 \
7 \ 6)
$$has length $5$, and the parity of $5 - 1 = 4$ is even, and so we
write $\sgn{\sigma} = +1$. On the other hand, $$\tau = (1 \ 3 \ 4
\ 7 \ 6 \ 5)   $$has length $6$, and the parity of $6 - 1 = 5$ is
odd, and so we write $\sgn{\tau} = -1$.

\bigskip

Recall also that if $(\sigma, \tau)\in S_n ^2$, then
$$ \sgn{\tau\sigma} = \sgn{\tau}\sgn{\sigma}. $$ Thus from the
above two examples
$$\sigma\tau = (1 \ 3 \ 4 \
7 \ 6)(1 \ 3 \ 4 \ 7 \ 6\ 5)
$$has signum $\sgn{\sigma} \sgn{\tau} = (+1)(-1) = -1$. Observe in
particular that for the identity permutation $\idefun\in S_n$ we
have $\sgn{\idefun} = +1$.

\begin{df}
Let $A\in\mat{n\times n}{ \bbR}, A = [a_{ij}]$ be a square matrix.
The \negrito{determinant of $A$} is defined and denoted by the  sum $$
\det A = \sum _{\sigma \in S_n} \sgn{\sigma} a_{1\sigma
(1)}a_{2\sigma (2)} \cdots a_{n\sigma (n)}.$$ \index{determinant}
\end{df}
\begin{rem}
The determinantal sum has $n!$ summands.
\end{rem}
\begin{exa}
If $n = 1$, then $S_1$ has only one member, $\idefun$, where
$\idefun (1) = 1$. Since $\idefun$ is an even permutation, $\sgn{
\idefun} = (+1)$ Thus if $A = (a_{11})$, then
$$\det A
 = a_{11}$$.
\end{exa}
\begin{exa}
If $n = 2$, then  $S_2$ has $2! = 2$   members,  $\idefun$ and
$\sigma =
 (1 \ 2)$. Observe that $\sgn{\sigma} = -1$. Thus if $$A = \begin{bmatrix}a_{11} & a_{12} \cr a_{21} & a_{22}\cr
 \end{bmatrix}$$ then
 $$\det A  = \sgn{\idefun}a_{1\idefun (1)}a_{2\idefun (2)} + \sgn{\sigma}a_{1\sigma (1)}a_{2\sigma (2)} = a_{11}a_{22} - a_{12}a_{21}.    $$
\end{exa}
\begin{exa}
From the above formula for $2\times 2$ matrices it follows that
$$\begin{array}{lll}\det A & = & \det \begin{bmatrix} 1 & 2 \cr 3 & 4 \cr   \end{bmatrix} \\ & = &  (1)(4) - (3)(2) = -2,\end{array} $$
$$\begin{array}{lll}\det B & = & \det \begin{bmatrix} -1 & 2 \cr 3 & 4 \cr
\end{bmatrix}(-1)(4) - (3)(2)\\ &  = &  -10, \end{array}$$ and $$ \det (A + B) = \det \begin{bmatrix} 0 & 4 \cr 6 & 8 \cr   \end{bmatrix}
 = (0)(8) - (6)(4) = -24.$$Observe in particular that $\det (A + B) \neq \det A + \det B$.
\end{exa}
\begin{exa}
If $n = 3$, then  $S_2$ has $3! = 6$   members:

$$\idefun, \tau_1 = (2 \ 3), \tau_2 = (1 \ 3), \tau_3 = (1 \ 2), \sigma_1 =  (1 \ 2 \ 3), \sigma_2 =  (1 \ 3 \ 2). $$.
Observe that $\idefun, \sigma_1, \sigma_2$ are even, and $\tau_1,
\tau_2, \tau_3$ are odd. Thus if $$A = \begin{bmatrix}a_{11} &
a_{12} & a_{13} \cr a_{21} & a_{22} & a_{23}\cr a_{31} & a_{32} &
a_{33}\cr
\end{bmatrix}$$then
 $$\begin{array}{lll}\det A  &  =  & \sgn{\idefun}a_{1\idefun (1)}a_{2\idefun (2)}a_{3\idefun (3)} + \sgn{\tau_1}a_{1\tau_1 (1)}a_{2\tau_1 (2)}a_{3\tau_1
 (3)}\\
& & \quad + \sgn{\tau_2}a_{1\tau_2 (1)}a_{2\tau_2 (2)}a_{3\tau_2
(3)} + \sgn{\tau_3}a_{1\tau_3 (1)}a_{2\tau_3 (2)}a_{3\tau_3 (3)}
\\
& & \quad + \sgn{\sigma_1}a_{1\sigma_1 (1)}a_{2\sigma_1
(2)}a_{3\sigma_1 (3)} + \sgn{\sigma_2}a_{1\sigma_2
(1)}a_{2\sigma_2 (2)}a_{3\sigma_2 (3)}
\\
&  = &  a_{11}a_{22}a_{33} - a_{11}a_{23}a_{32}  - a_{13}a_{22}a_{31}\\
& & \qquad - a_{13}a_{21}a_{33}  + a_{12}a_{23}a_{31} + a_{13}a_{21}a_{32}. \\
\end{array}
$$
\end{exa}
\begin{thm}[Row-Alternancy of Determinants]\label{thm:alternating_determinants}
Let $A \in \mat{n\times n}{ \bbR}, A = [a_{ij}]$. If $B \in
\mat{n\times n}{ \bbR}, B = [b_{ij}]$ is the matrix obtained by
interchanging  the $s$-th row of $A$ with its $t$-th row, then $\det
B = -\det A$.
\end{thm}
\begin{proof}
Let $\tau$ be the transposition
$$\tau =
\begin{bmatrix} s & t \cr \tau (t) & \tau (s) \cr
\end{bmatrix}.
$$Then $\sigma\tau (a) = \sigma (a)$ for $a\in\{1, 2, \ldots , n\}\setminus
\{s,t\}$. Also, $\sgn{\sigma\tau} = \sgn{\sigma}\sgn{\tau} =
-\sgn{\sigma}$.  As $\sigma$ ranges through all permutations of
$S_n$, so does
 $\sigma\tau$, hence
$$\begin{array}{lll}\det B & = & \sum _{\sigma\in S_n}\sgn{\sigma} b_{1\sigma (1)}b_{2\sigma (2)}\cdots
b_{s\sigma (s)}\cdots b_{t\sigma (t)} \cdots b_{n\sigma (n)} \\ &
= &  \sum _{\sigma\in S_n}\sgn{\sigma} a_{1\sigma (1)}a_{2\sigma
(2)}\cdots a_{st}\cdots a_{ts} \cdots a_{n\sigma (n)} \\
& = & -\sum _{\sigma\in S_n}\sgn{\sigma\tau }a_{1\sigma\tau
(1)}a_{2\sigma\tau (2)}\cdots a_{s\sigma\tau
(s)}\cdots a_{t\sigma\tau  (t)} \cdots a_{n\sigma\tau (n)} \\
& = & -\sum _{\lambda \in S_n} \sgn{\lambda } a_{1\lambda
(1)}a_{2\lambda  (2)} \cdots a_{n\lambda  (n)} \\
& = & -\det A.
 \end{array} $$
\end{proof}
\begin{cor}If $A_{(r:k)}, 1 \leq k \leq n$ denote the rows of $A$
and $\sigma \in S_n$, then $$\det \begin{bmatrix} A_{(r:\sigma
(1))} \cr A_{(r:\sigma (2))} \cr \vdots \cr A_{(r:\sigma (n))}\cr
\end{bmatrix} = (\sgn{\sigma})\det A. $$An analogous result holds
for columns. \label{cor:alternating_determinants}\end{cor}
\begin{proof}
Apply the result of Theorem \ref{thm:alternating_determinants}
multiple times.
\end{proof}
\begin{thm} \label{thm:determinant_of_transpose}
Let $A \in \mat{n\times n}{ \bbR}, A = [a_{ij}]$. Then $$\det A^T =
\det A.$$
\end{thm}
\begin{proof}
Let $C = A^T$. By definition $$\begin{array}{lll}\det A^T & = &
\det C \\ & = &  \sum _{\sigma \in S_n} \sgn{\sigma} c_{1\sigma
(1)}c_{2\sigma (2)} \cdots c_{n\sigma (n)} \\ &   = &  \sum
_{\sigma \in S_n} \sgn{\sigma} a_{\sigma (1)1}a_{\sigma (2)2}
\cdots a_{\sigma (n)n}. \end{array}$$But the product $a_{\sigma
(1)1}a_{\sigma (2)2} \cdots a_{\sigma (n)n}$ also appears in $\det
A$ with the same signum $\sgn{\sigma}$, since the permutation
$$  \begin{bmatrix}\sigma (1) & \sigma (2) & \cdots & \sigma (n)\cr 1 & 2 & \cdots & n \cr
\end{bmatrix}$$is the inverse of the permutation
$$  \begin{bmatrix}1 & 2 & \cdots & n \cr \sigma (1) & \sigma (2) & \cdots & \sigma (n)\cr
\end{bmatrix}.$$
\end{proof}
\begin{cor}[Column-Alternancy of Determinants]\label{cor:alternating_determinants_1}
Let $A \in \mat{n\times n}{ \bbR}, A = [a_{ij}]$. If $C \in
\mat{n\times n}{ \bbR}, C = [c_{ij}]$ is the matrix obtained by
interchanging the $s$-th column of $A$ with its $t$-th column, then
$\det C = -\det A$.
\end{cor}
\begin{proof}
This follows upon combining Theorem
\ref{thm:alternating_determinants} and Theorem
\ref{thm:determinant_of_transpose}.
\end{proof}
\begin{thm}[Row Homogeneity of Determinants]\label{thm:row_homogeneity_of_determinants}
Let $A \in \mat{n\times n}{ \bbR}, A = [a_{ij}]$ and $\alpha\in
\bbR$. If $B \in \mat{n\times n}{ \bbR}, B = [b_{ij}]$ is the
matrix obtained by multiplying the $s$-th row of $A$ by $\alpha$,
then
$$\det B = \alpha \det A.
$$
\end{thm}
\begin{proof}
Simply observe that $$\sgn{\sigma} a_{1\sigma (1)}a_{2\sigma
(2)}\cdots \alpha a_{s\sigma (s)} \cdots a_{n\sigma (n)} = \alpha
\sgn{\sigma} a_{1\sigma (1)}a_{2\sigma (2)}\cdots  a_{s\sigma (s)}
\cdots a_{n\sigma (n)}.
$$

\end{proof}
\begin{cor}[Column Homogeneity of Determinants]\label{cor:column_homogeneity_of_determinants}
If $C \in \mat{n\times n}{ \bbR}, C = (C_{ij})$ is the matrix
obtained by multiplying the $s$-th column of $A$ by $\alpha$, then
$$\det C = \alpha \det A.
$$
\end{cor}
\begin{proof}
This follows upon using Theorem \ref{thm:determinant_of_transpose}
and Theorem \ref{thm:row_homogeneity_of_determinants}.
\end{proof}
\begin{rem}
It follows from Theorem \ref{thm:row_homogeneity_of_determinants}
and Corollary \ref{cor:column_homogeneity_of_determinants} that if a
row (or column) of a matrix consists of $0_{\bbR}$s only, then the
determinant of this matrix is $0_{\bbR}$.
\end{rem}
\begin{exa}
$$\det
\begin{bmatrix} x & 1 & a \cr x^2 & 1 & b \cr x^3 & 1 & c
\cr\end{bmatrix} = x\det\begin{bmatrix} 1 & 1 & a \cr x & 1 & b
\cr x^2 & 1 & c \cr\end{bmatrix}. $$
\end{exa}

\begin{cor}
$$\det (\alpha A) = \alpha ^n\det A.$$
\end{cor}
\begin{proof}
Since there are $n$ columns, we are able to pull out one factor of
$\alpha$ from each one.
\end{proof}
\begin{exa}
Recall that a matrix $A$ is \negrito{skew-symmetric} if $A =
 -A^T$. Let $A \in\mat{2001}{\bbR}$ be
 skew-symmetric. Prove that $\det A = 0.$
 \end{exa}\begin{solu}We have $$\det A = \det (-A^T) = (-1)^{2001}\det A^T = -\det A,$$
 and so $2\det A = 0,$ from where $\det A = 0$.
 \end{solu}
 \begin{lemma}[Row-Linearity and Column-Linearity of Determinants]
Let $A \in \mat{n\times n}{ \bbR}, A = [a_{ij}]$. For a fixed row
$s$, suppose that $a_{sj} = b_{sj} + c_{sj}$ for each $j \in [1;n]$.
Then
$$\begin{array}{l}\tiny{\det \begin{bmatrix} a_{11}&  a_{12}&  \cdots &
a_{1n} \cr a_{21}&  a_{22}&  \cdots &  a_{2n} \cr \vdots & \vdots
& \cdots & \vdots & \vdots \cr a_{(s - 1)1}& a_{(s - 1)2}& \cdots
& a_{(s -1)n} \cr b_{s1} + c_{s1}& b_{s2} +c_{s2}& \cdots & b_{sn}
+c_{sn} \cr a_{(s+1)1}& a_{(s+1)2}& \cdots & a_{(s+1)n} \cr
\vdots & \vdots & \cdots & \vdots & \vdots \cr a_{n1}& a_{n2}&
\cdots & a_{nn} \cr
\end{bmatrix} } \vspace{2mm}\\   \tiny{=  \det   \begin{bmatrix} a_{11}&  a_{12}&  \cdots &  a_{1n} \cr
a_{21}&  a_{22}&  \cdots &  a_{2n} \cr \vdots & \vdots & \cdots &
\vdots & \vdots \cr a_{(s - 1)1}& a_{(s - 1)2}& \cdots & a_{(s
-1)n} \cr b_{s1}& b_{s2}& \cdots & b_{sn} \cr a_{(s+1)1}&
a_{(s+1)2}& \cdots & a_{(s+1)n} \cr  \vdots & \vdots & \cdots &
\vdots & \vdots \cr a_{n1}& a_{n2}& \cdots &  a_{nn} \cr
\end{bmatrix}}\vspace{2mm}\\ \qquad  \tiny{+ \det \begin{bmatrix} a_{11}&  a_{12}&  \cdots & a_{1n}
\cr a_{21}&  a_{22}&  \cdots &  a_{2n} \cr \vdots & \vdots &
\cdots & \vdots & \vdots \cr a_{(s - 1)1}& a_{(s - 1)2}& \cdots &
a_{(s -1)n} \cr c_{s1}& c_{s2}& \cdots & c_{sn} \cr a_{(s + 1)1}&
a_{(s+1)2}& \cdots & a_{(s+1)n} \cr  \vdots & \vdots & \cdots &
\vdots & \vdots \cr a_{n1}& a_{n2}& \cdots &  a_{nn} \cr
\end{bmatrix}}.
 \end{array} $$
 \label{lem:row_linearity_of_determinants}
An analogous result holds for columns.



 \end{lemma}
\begin{proof}
Put $$\tiny{S =  \begin{bmatrix} a_{11}&  a_{12}&  \cdots & a_{1n}
\cr a_{21}&  a_{22}&  \cdots &  a_{2n} \cr \vdots & \vdots &
\cdots & \vdots & \vdots \cr a_{(s - 1)1}& a_{(s - 1)2}& \cdots &
a_{(s -1)n} \cr b_{s1} + c_{s1}& b_{s2} +c_{s2}& \cdots & b_{sn}
+c_{sn} \cr a_{(s+1)1}& a_{(s+1)2}& \cdots & a_{(s+1)n} \cr \vdots
& \vdots & \cdots & \vdots & \vdots \cr a_{n1}& a_{n2}& \cdots &
a_{nn} \cr
\end{bmatrix},} $$
$$\tiny{T =  \begin{bmatrix} a_{11}&  a_{12}&  \cdots &  a_{1n} \cr
a_{21}&  a_{22}&  \cdots &  a_{2n} \cr \vdots & \vdots & \cdots &
\vdots & \vdots \cr a_{(s - 1)1}& a_{(s - 1)2}& \cdots & a_{(s
-1)n} \cr b_{s1}& b_{s2}& \cdots & b_{sn} \cr a_{(s+1)1}&
a_{(s+1)2}& \cdots & a_{(s+1)n} \cr  \vdots & \vdots & \cdots &
\vdots & \vdots \cr a_{n1}& a_{n2}& \cdots &  a_{nn} \cr
\end{bmatrix}} $$ and $$\tiny{U = \begin{bmatrix} a_{11}&  a_{12}&  \cdots & a_{1n}
\cr a_{21}&  a_{22}&  \cdots &  a_{2n} \cr \vdots & \vdots &
\cdots & \vdots & \vdots \cr a_{(s - 1)1}& a_{(s - 1)2}& \cdots &
a_{(s -1)n} \cr c_{s1}& c_{s2}& \cdots & c_{sn} \cr a_{(s + 1)1}&
a_{(s+1)2}& \cdots & a_{(s+1)n} \cr  \vdots & \vdots & \cdots &
\vdots & \vdots \cr a_{n1}& a_{n2}& \cdots &  a_{nn} \cr
\end{bmatrix}.}  $$Then
$${\footnotesize \begin{array}{lll}\det S &  = & \sum _{\sigma\in S_n}\sgn{\sigma} a_{1\sigma (1)}a_{2\sigma (2)}\cdots a_{(s-1)\sigma (s - 1)}(b_{s\sigma
(s)}\\ & & \qquad +  c_{s\sigma (s)})a_{(s+1)\sigma (s + 1)}
\cdots a_{n\sigma (n)} \\ & = &
 \sum _{\sigma\in S_n}\sgn{\sigma} a_{1\sigma (1)}a_{2\sigma (2)}\cdots a_{(s-1)\sigma (s - 1)}b_{s\sigma
(s)}a_{(s+1)\sigma (s + 1)}   \cdots a_{n\sigma (n)} \\ & &  +
\sum _{\sigma\in S_n}\sgn{\sigma} a_{1\sigma (1)}a_{2\sigma
(2)}\cdots a_{(s-1)\sigma (s - 1)}c_{s\sigma
(s)}a_{(s+1)\sigma (s + 1)}   \cdots a_{n\sigma (n)}  \\
& = & \det T + \det U.
\end{array} } $$
By applying the above argument to $A^T$, we obtain the result for
columns.

\end{proof}

\begin{lemma}
If two rows or two columns of $A \in \mat{n\times n}{ \bbR}, A =
[a_{ij}]$ are identical, then $\det A = 0_{\bbR}$.
\label{lem:determinant_with_two_identical_rows}\end{lemma}
\begin{proof}
Suppose $a_{sj} = a_{tj}$ for $s\neq t$ and for all $j\in [1;n]$.
In particular, this means that for any $\sigma\in S_n$ we have
$a_{s\sigma (t)} = a_{t\sigma (t)}$ and $a_{t\sigma (s)} =
a_{s\sigma (s)}$.  Let $\tau$ be the transposition
$$\tau =
\begin{bmatrix} s & t \cr \tau (t) & \tau (s) \cr
\end{bmatrix}.
$$Then $\sigma\tau (a) = \sigma (a)$ for $a\in\{1, 2, \ldots , n\}\setminus
\{s,t\}$. Also, $\sgn{\sigma\tau} = \sgn{\sigma}\sgn{\tau} =
-\sgn{\sigma}$. As $\sigma$ runs through all even permutations,
$\sigma\tau$ runs through all odd permutations, and viceversa.
Therefore
$$\begin{array}{lll}det A & = & \sum _{\sigma \in S_n}\sgn{\sigma} a_{1\sigma (1)}a_{2\sigma (2)}\cdots a_{s\sigma
(s)}\cdots a_{t\sigma (t)} \cdots a_{n\sigma (n)} \\
& = &  \sum _{\stackrel{\sigma \in S_n}{\sgn{\sigma} =
1}}\left(\sgn{\sigma} a_{1\sigma (1)}a_{2\sigma (2)}\cdots
a_{s\sigma (s)}\cdots a_{t\sigma (t)} \cdots a_{n\sigma (n)}\right.\\
& & \left. \quad + \sgn{\sigma\tau} a_{1\sigma\tau
(1)}a_{2\sigma\tau (2)}\cdots a_{s\sigma\tau (s)}\cdots
a_{t\sigma\tau (t)} \cdots
a_{n\sigma\tau (n)}\right) \\
& = &  \sum _{\stackrel{\sigma \in S_n}{\sgn{\sigma} =
1}}\sgn{\sigma}\left( a_{1\sigma (1)}a_{2\sigma (2)}\cdots
a_{s\sigma (s)}\cdots a_{t\sigma (t)} \cdots a_{n\sigma (n)}\right.\\
& & \left. \quad -  a_{1\sigma (1)}a_{2\sigma (2)}\cdots
a_{s\sigma (t)}\cdots a_{t\sigma (s)} \cdots
a_{n\sigma (n)}\right) \\
& = &  \sum _{\stackrel{\sigma \in S_n}{\sgn{\sigma} =
1}}\sgn{\sigma}\left( a_{1\sigma (1)}a_{2\sigma (2)}\cdots
a_{s\sigma (s)}\cdots a_{t\sigma (t)} \cdots a_{n\sigma (n)}\right.\\
& & \left. \quad -  a_{1\sigma (1)}a_{2\sigma (2)}\cdots
a_{t\sigma (t)}\cdots a_{s\sigma (s)} \cdots
a_{n\sigma (n)}\right) \\
& = & 0_{\bbR}.
\end{array}$$
Arguing on $A^T$ will yield the analogous result for the columns.
\end{proof}
\begin{cor}
If two rows or two columns of $A \in \mat{n\times n}{ \bbR}, A =
[a_{ij}]$ are proportional, then $\det A = 0_{\bbR}$.
\label{cor:determinant_with_two_proportional_rows}\end{cor}
\begin{proof}
Suppose $a_{sj} = \alpha a_{tj}$ for $s\neq t$ and for all $j\in
[1;n]$. If $B$ is the matrix obtained by pulling out the factor
$\alpha$ from the $s$-th row then $\det A = \alpha \det B$. But now
the $s$-th and the $t$-th rows in $B$ are identical, and so $\det B
= 0_{\bbR}$. Arguing on $A^T$ will yield the analogous result for
the columns.

\end{proof}
\begin{exa}
$$\det \begin{bmatrix} 1 & a & b \cr 1 & a & c \cr 1 & a & d \cr   \end{bmatrix} =
a\det \begin{bmatrix} 1 & 1 & b \cr 1 & 1 & c \cr 1 & 1 & d \cr
\end{bmatrix} = 0,  $$since on the last determinant the first two columns are
identical.

\end{exa}

 \begin{thm}[Multilinearity of Determinants]\label{thm:multilinearity_of_determinants}
Let $A \in \mat{n\times n}{ \bbR}, A = [a_{ij}]$ and $\alpha\in
\bbR$. If $X \in \mat{n\times n}{ \bbR}, X = (x_{ij})$ is the
matrix obtained by the row transvection  $R_s + \alpha R_t
\rightarrow R_s$ then $\det X = \det A$. Similarly, if $Y \in
\mat{n\times n}{ \bbR}, Y = (y_{ij})$ is the matrix obtained by the
column transvection  $C_s + \alpha C_t \rightarrow C_s$ then $\det Y
= \det A$.
 \end{thm}
 \begin{proof}
For the row transvection it suffices to take $b_{sj} = a_{sj}$,
$c_{sj} = \alpha a_{tj}$ for $j\in [1;n]$ in Lemma
\ref{lem:row_linearity_of_determinants}. With the same notation as
in the lemma, $T = A$, and so, $$\det X = \det T + \det U = \det A
+ \det U.
$$But $U$ has its $s$-th and $t$-th rows proportional ($s\neq t$),
and so by Corollary \ref{cor:determinant_with_two_proportional_rows}
$\det U = 0_{\bbR}.$ Hence $\det X = \det A.$ To obtain the result
for column transvections it suffices now to also apply Theorem
\ref{thm:determinant_of_transpose}.
 \end{proof}


\begin{exa}
Demonstrate, {\em without actually calculating the determinant}
that $$\det\begin{bmatrix}2 & 9 & 9 \cr 4 & 6 & 8 \cr 7 & 4 & 1
\cr
\end{bmatrix}$$is divisible by $13$.
\end{exa}
\begin{solu}Observe that $299, 468$ and $741$ are all divisible by
13. Thus $$\det\begin{bmatrix}2 & 9 & 9 \cr 4 & 6 & 8 \cr 7 & 4 &
1 \cr \end{bmatrix} \stackrel{C_3 + 10C_2 + 100C_1 \rightarrow
C_3}{ = }\det\begin{bmatrix}2 & 9 & 299 \cr 4 & 6 & 468 \cr 7 & 4
& 741 \cr \end{bmatrix} = 13\det\begin{bmatrix}2 & 9 & 23 \cr 4 &
6 & 36 \cr 7 & 4 & 57 \cr \end{bmatrix},$$which shews that the
determinant is divisible by $13$.
\end{solu}

 \begin{thm}
The determinant of a triangular matrix (upper or lower) is the
product of its diagonal elements.
 \end{thm}
 \begin{proof}
Let $A\in\mat{n\times n}{ \bbR}, A = [a_{ij}]$ be a triangular
matrix. Observe that if $\sigma \neq \idefun$ then $a_{i\sigma
(i)}a_{\sigma(i)\sigma^2(i)} = 0_{\bbR}$ occurs in the product
$$a_{1\sigma (1)}a_{2\sigma (2)}
\cdots a_{n\sigma (n)}.  $$Thus $$\begin{array}{lll}\det A & = &
\sum _{\sigma \in S_n}\sgn{\sigma}a_{1\sigma (1)}a_{2\sigma
(2)}\cdots a_{n\sigma (n)}\\ &  = &  \sgn{\idefun}a_{1\idefun
(1)}a_{2\idefun (2)}\cdots a_{n\idefun (n)} = a_{11}a_{22} \cdots
a_{nn}. \end{array}
$$
 \end{proof}
 \begin{exa}
The determinant of the $n\times n$ identity matrix ${\textbf  I}_n$ over
a field $\bbR$ is $$\det  {\textbf  I}_n = 1_{\bbR}. $$
 \end{exa}

\begin{exa}
Find $$\det\begin{bmatrix} 1 & 2 & 3 \cr 4 & 5 & 6 \cr  7 & 8 & 9
\cr
\end{bmatrix}.$$
\end{exa}
\begin{solu}We have $$\begin{array}{lll}\det\begin{bmatrix} 1 & 2 &
3 \cr 4 & 5 & 6 \cr  7 & 8 & 9 \cr
\end{bmatrix} & \grstep{\stackrel{C_2 - 2C_1 \rightarrow C_2}{C_3 - 3C_1 \rightarrow C_3}}
& \det\begin{bmatrix} 1 & 0 & 0 \cr 4 & -3 & -6 \cr  7 & -6 &
-12\cr
\end{bmatrix} \\
& = & (-3)(-6)\det\begin{bmatrix} 1 & 0 & 0 \cr 4 & 1 & 1 \cr  7 &
2 & 2\cr
\end{bmatrix} \\
& = & 0,
\end{array}$$
since in this last matrix the second and third columns are
identical and so Lemma
\ref{lem:determinant_with_two_identical_rows} applies.
\end{solu}
\begin{thm}
Let $(A, B)\in (\mat{n\times n}{ \bbR})^2$. Then $$\det (AB) =
(\det A)(\det B).
$$
\label{thm:product_of_determinants}\end{thm}
\begin{proof}
Put $D = AB, D = (d_{ij}), d_{ij} = \sum _{k = 1} ^n
a_{ik}b_{kj}$. If $A_{(c:k)}, D_{(c:k)}, 1 \leq k \leq n$ denote
the columns of $A$ and $D$, respectively, observe that $$
D_{(c:k)} = \sum _{l = 1} ^n b_{lk}A_{(c:l)}, \    \  1 \leq k
\leq n.
$$
Applying Corollary \ref{cor:column_homogeneity_of_determinants}
and Lemma \ref{lem:row_linearity_of_determinants} multiple times,
we obtain
$$\begin{array}{lll}\det D  & =  & \det (D_{(c:1)},  D_{(c:2)}, \ldots , D_{(c:n)}) \\ &
=&  \sum _{j_1 = 1} ^n\sum _{j_2 = 1} ^n \cdots \sum _{j_n = 1} ^n
b_{1j_1}b_{2j_2}\cdots b_{nj_n}\\ & & \quad \cdot\det
(A_{(c:j_1)}, A_{(c:j_2)}, \ldots , A_{(c:j_n)}). \end{array}
$$
By Lemma \ref{lem:determinant_with_two_identical_rows}, if any two
of the $A_{(c:j_l)}$ are identical, the determinant on the right
vanishes. So each one of the $j_l$ is different in the
non-vanishing terms and so the map
$$\fun{\sigma}{l}{j_l}{\{1, 2, \ldots , n\}}{\{1, 2, \ldots , n\}}
$$is a permutation. Here $j_l = \sigma (l)$. Therefore, for the non-vanishing $$\det (A_{(c:j_1)},  A_{(c:j_2)},
\ldots , A_{(c:j_n)})$$ we have in view of Corollary
\ref{cor:alternating_determinants},  $$\begin{array}{lll}\det
(A_{(c:j_1)}, A_{(c:j_2)}, \ldots , A_{(c:j_n)}) & = &
(\sgn{\sigma})\det (A_{(c:1)}, A_{(c:2)}, \ldots , A_{(c:n)}) \\ &
= &(\sgn{\sigma})\det A.\end{array}$$We deduce that
$$\begin{array}{lll}\det (AB) & = & \det D \\ & = & \sum _{j_n = 1} ^n
b_{1j_1}b_{2j_2}\cdots b_{nj_n}\det (A_{(c:j_1)},  A_{(c:j_2)},
\ldots , A_{(c:j_n)})\\
&  = & (\det A) \sum _{\sigma\in S_n} (\sgn{\sigma}) b_{1\sigma
(1)}b_{2\sigma (2)}\cdots b_{n\sigma (n)}\\
& = & (\det A)(\det B),
\end{array}$$as we wanted to shew.
\end{proof}
By applying the preceding theorem multiple times we obtain
\begin{cor}
If $A\in\mat{n\times n}{ \bbR}$ and if $k$ is a positive integer
then $$ \det A^k = (\det A)^k.
$$
\end{cor}
\begin{cor}
If $A\in\gl{n}{ \bbR}$ and if $k$ is a positive integer then $\det
A \neq 0_{\bbR}$ and
$$ \det A^{-k} = (\det A)^{-k}.
$$
\end{cor}
\begin{proof}
We have $AA^{-1} = {\textbf  I}_n$ and so by Theorem
\ref{thm:product_of_determinants} $(\det A)(\det A^{-1}) = 1_{\bbR
}$, from where the result follows.
\end{proof}

\section*{\psframebox{Homework}}
\begin{pro}
 Let $$\Omega =
\det
\begin{bmatrix} bc & ca & ab \cr a & b & c \cr a^2 & b^2 & c^2 \cr\end{bmatrix}
.
$$Without expanding either determinant, prove that
$$\Omega = \det
\begin{bmatrix} 1 & 1 & 1 \cr a^2 & b^2 & c^2 \cr a^3 & b^3 & c^3
\cr\end{bmatrix}.
$$
\begin{answer} Multiplying the first column of the given matrix  by $a$,
its second column by $b$, and its third column by $c$, we obtain
$$abc\Omega =  \begin{bmatrix} abc & abc & abc \cr a^2 & b^2 & c^2 \cr a^3 & b^3 & c^3 \cr\end{bmatrix}. $$ We may factor out
$abc$ from the first row of this last matrix thereby obtaining
$$abc\Omega = abc\det
\begin{bmatrix} 1 & 1 & 1 \cr a^2 & b^2 & c^2 \cr a^3 & b^3 & c^3
\cr\end{bmatrix}. $$Upon dividing by $abc$,
$$\Omega = \det
\begin{bmatrix} 1 & 1 & 1 \cr a^2 & b^2 & c^2 \cr a^3 & b^3 & c^3
\cr\end{bmatrix}.
$$
\end{answer}
\end{pro}
\begin{pro}
Demonstrate that $$\Omega = \det\begin{bmatrix} a - b - c & 2a &
2a \cr 2b & b - c - a & 2b \cr 2c & 2c & c - a - b \cr
\end{bmatrix} = (a + b + c)^3.
$$
\begin{answer} Performing $R_1 + R_2 + R_3 \rightarrow R_1$ we have
$$\begin{array}{l}
\Omega = \det\begin{bmatrix} a - b - c & 2a & 2a \cr 2b & b - c -
a & 2b \cr 2c & 2c & c - a - b \cr
\end{bmatrix}\vspace{2mm}\\ \qquad = \det\begin{bmatrix} a + b + c & a + b + c & a + b + c \cr 2b
& b - c - a & 2b \cr 2c & 2c & c - a - b \cr
\end{bmatrix}.\end{array}$$ Factorising $(a + b + c)$ from the first row of
this last determinant, we have
$$\Omega =   (a + b + c)\det\begin{bmatrix} 1 & 1 & 1 \cr 2b
& b - c - a & 2b \cr 2c & 2c & c - a - b \cr
\end{bmatrix}. $$Performing $C_2 - C_1 \rightarrow C_2$ and $C_3 - C_1 \rightarrow
C_3$,
$$\Omega =   (a + b + c)\det\begin{bmatrix} 1 & 0 & 0 \cr 2b
& -b - c - a & 0 \cr 2c & 0 & -c - a - b \cr
\end{bmatrix}. $$This last matrix is triangular, hence $$\Omega = (a + b + c)(-b - c - a)(-c -a - b) = (a + b + c)^3,  $$
as wanted.
\end{answer}
\end{pro}

\begin{pro} After the indicated column operations on a $3\times 3$
 matrix $A$ with $\det A = -540$, matrices $A_1, A_2, \ldots , A_5$
 are successively obtained:
 $$A \stackrel{C_1 + 3C_2 \rightarrow C_1}{\rightarrow} A_1
 \stackrel{C_2 \leftrightarrow C_3}{\rightarrow} A_2 \stackrel{3C_2
 - C_1 \rightarrow C_2}{\rightarrow} A_3 \stackrel{C_1 - 3C_2
 \rightarrow C_1}{\rightarrow} A_4 \stackrel{2C_1  \rightarrow
 C_1}{\rightarrow} A_5$$ Determine  the numerical values
 of $\det A_1, \det A_2, \det A_3, \det A_4$ and $\det A_5.$
 \begin{answer} $\det A_1 = \det A = -540$ by multilinearity. $\det A_2 = -\det
 A_1 = 540$ by alternancy. $\det A_3 = 3\det A_2 = 1620$  by both
 multilinearity and homogeneity from one column. $\det A_4 = \det A_3
 = 1620$ by multilinearity, and $\det A_5 = 2\det A_4 = 3240$ by
 homogeneity from one column.
 \end{answer}
\end{pro}

\begin{pro}
Prove, without actually expanding the determinant, that
$$\det \begin{bmatrix}1 & 2 & 3 & 7 & 0 \cr 6 & 1 & 5 & 14 & 1 \cr 8 & 6 & 1 & 21 & 3 \cr 7 & 3 & 8 & 7 & 1 \cr 2 & 4 & 6 & 0 & 4 \cr  \end{bmatrix}$$
is divisible by $1722$.
\end{pro}


\begin{pro} Let $A, B, C$ be $3\times 3$ matrices with $\det A = 3, \det
B^3 = -8,
 \det C = 2$. Compute (i) $\det ABC$, (ii) $\det 5AC$, (iii) $\det
 A^3B^{-3}C^{-1}$. Express your answers as fractions. \begin{answer} From the given data, $\det B = -2.$ Hence
 $$\det ABC = (\det A)(\det B)(\det C) = -12,$$
 $$\det 5AC = 5^3\det AC = (125)(\det A)(\det C) = 750,$$
 $$(\det A^3B^{-3}C^{-1}) = \frac{(\det A)^3}{(\det B)^3(\det C)} = -\frac{27}{16}.$$
\end{answer}
\end{pro}
\begin{pro}
Shew that $\forall A\in\mat{n\times n}{\bbR},$ $$\exists (X, Y)\in
(\mat{n\times n}{\bbR})^2, (\det X)(\det Y)\neq 0$$ such that
$$A = X + Y.$$ That is, any square matrix over $\bbR$ can be written as a sum of two matrices whose determinant is not zero.\begin{answer} Pick $\lambda \in \bbR \setminus \{0,
a_{11}, a_{22}, \ldots , a_{nn}\}$. Put
$$X = \begin{bmatrix} a_{11} - \lambda & 0 & 0 & \cdots &  0 \cr a_{21}  & a_{22} - \lambda & 0 & \cdots & 0 \cr
a_{31}  & a_{32}  & a_{33} - \lambda & \cdots &  0 \cr \vdots &
\vdots & \vdots & \vdots & \vdots \cr a_{n1} & a_{n2} & a_{n3} &
\cdots & a_{nn} - \lambda
\end{bmatrix}$$and
$$Y = \begin{bmatrix}
\lambda & a_{12} & a_{13} & \vdots & a_{1n} \cr   0 & \lambda &
a_{23} & \vdots & a_{2n} \cr
 0 & 0 &
\lambda & \vdots & a_{3n} \cr \vdots & \vdots & \vdots & \vdots &
\vdots \cr
 0 & 0 &
0 & \vdots & \lambda \cr


\end{bmatrix}$$ Clearly  $A = X + Y$, $\det X = (a_{11} - \lambda)(a_{22} -
\lambda)\cdots (a_{nn} - \lambda) \neq 0$, and $\det Y = \lambda^n
\neq 0$. This completes the proof.
\end{answer}
\end{pro}
\begin{pro}
Prove or disprove! The set $X = \{A\in\mat{n\times n}{ \bbR}: \det
A = 0_{\bbR}\}$ is a vector subspace of $\mat{n\times n}{ \bbR}$.
\begin{answer}
No.
\end{answer}
\end{pro}
\section{Laplace Expansion} We now develop a more
computationally convenient approach to determinants.

\bigskip
Put $$C_{ij} = \sum _{\stackrel{\sigma \in S_n}{\sigma (i) = j}}
(\sgn{\sigma})a_{1\sigma (1)}a_{2\sigma (2)}\cdots a_{n\sigma
(n)}. $$Then
\begin{equation}\label{eq:cofactor}\begin{array}{lll}\det A & = &
\sum _{\sigma \in S_n} (\sgn{\sigma})a_{1\sigma (1)}a_{2\sigma
(2)}\cdots a_{n\sigma (n)}
\\ &  = &   \sum _{i = 1} ^n a_{ij}\sum _{\stackrel{\sigma \in
S_n}{\sigma (i) = j}} (\sgn{\sigma})a_{1\sigma (1)}a_{2\sigma
(2)}\\ & &  \qquad \cdots a_{(i - 1)\sigma (i - 1)}a_{(i +
1)\sigma (i + 1)}\cdots a_{n\sigma (n)} \\& = &  \sum _{i = 1} ^n
a_{ij}C_{ij},
\end{array}\end{equation}is the expansion of $\det A$ along the $j$-th column.
Similarly,
$$\begin{array}{lll}\det A & = &  \sum _{\sigma \in S_n}
(\sgn{\sigma})a_{1\sigma (1)}a_{2\sigma (2)}\cdots a_{n\sigma (n)}
\\ & = &   \sum _{j = 1} ^n a_{ij}\sum _{\stackrel{\sigma \in
S_n}{\sigma (i) = j}} (\sgn{\sigma})a_{1\sigma (1)}a_{2\sigma
(2)}\\ & & \qquad \cdots a_{(i - 1)\sigma (i - 1)}a_{(i + 1)\sigma
(i + 1)}\cdots a_{n\sigma (n)}\\ & = & \sum _{j = 1} ^n
a_{ij}C_{ij},
\end{array}$$is the expansion of $\det A$ along the $i$-th row.
\begin{df}
Let $A\in\mat{n\times n}{ \bbR}, A = [a_{ij}]$. The $ij$-th minor
$A_{ij} \in\mat{n - 1}{\bbR}$ is the $(n - 1)\times (n - 1)$  matrix
obtained by deleting the $i$-th row and the $j$-th column from $A$.
\end{df}
\begin{exa}
If $$ A = \begin{bmatrix} 1 & 2 & 3 \cr 4 & 5 & 6 \cr  7 & 8 & 9
\cr
\end{bmatrix}$$then, for example,  $$A_{11} =\begin{bmatrix} 5 & 6 \cr 8 & 9 \cr \end{bmatrix}, \
\ \ A_{12} = \begin{bmatrix} 4 & 6 \cr 7 & 9 \cr  \end{bmatrix}, \
\ \ A_{21} = \begin{bmatrix} 2 & 3 \cr 8 & 9 \cr  \end{bmatrix}, \
\ \ A_{22} = \begin{bmatrix} 1 & 3 \cr 7 & 9 \cr  \end{bmatrix}, \
\ \ A_{33} = \begin{bmatrix} 1 & 2 \cr 4 & 5 \cr  \end{bmatrix}.
$$
\end{exa}
\begin{thm}\label{thm:laplace_expansion}
Let $A\in\mat{n\times n}{ \bbR}$. Then $$\det A = \sum _{i = 1} ^n
a_{ij}(-1)^{i + j}\det A_{ij} = \sum _{j = 1} ^n a_{ij}(-1)^{i +
j}\det A_{ij}.
$$
\end{thm}
\begin{proof}
It is enough to shew, in view of  \ref{eq:cofactor} that
$$ (-1)^{i + j}\det A_{ij} =  C_{ij}. $$Now,
$$\begin{array}{lll}C_{nn} & = & \sum _{\stackrel{\sigma\in S_n}{\sigma (n) = n}} \sgn{\sigma} a_{1\sigma (1)}a_{2\sigma (2)}\cdots a_{(n-1)\sigma (n-1)}
 \\ & = & \sum _{\tau\in S_{n-1}} \sgn{\tau} a_{1\tau (1)}a_{2\tau (2)}\cdots a_{(n - 1)\tau (n -1)} \\ & = & \det
A_{nn},\end{array}$$since the second sum shewn is the determinant
of the submatrix obtained by deleting the last row and last column
from $A$.

\bigskip


To find $C_{ij}$ for general $ij$ we perform some row and column
interchanges to $A$ in order to bring $a_{ij}$ to the $nn$-th
position. We thus bring the $i$-th row to the $n$-th row by a
series of transpositions, first swapping the $i$-th and the $(i +
1)$-th row, then swapping the new $(i + 1)$-th row and the $(i +
2)$-th row, and so forth until the original $i$-th row makes it to
the $n$-th row. We have made thereby $n - i$ interchanges. To this
new matrix we perform analogous interchanges to the $j$-th column,
thereby making $n - j$ interchanges. We have made a total of $2n -
i - j$ interchanges. Observe that $(-1)^{2n - i - j} = (-1)^{i +
j}$. Call the analogous quantities in the resulting matrix $A',
C'_{nn}, A'_{nn} $. Then
$$C_{ij} = C'_{nn} = \det A'_{nn} = (-1)^{i + j}\det A_{ij},
$$by virtue of Corollary \ref{cor:alternating_determinants}.

\end{proof}
\begin{rem}
It is irrelevant which row or column we choose to expand a
determinant of a square matrix. We always obtain the same result.
The sign pattern is given by $$\begin{bmatrix} + & - & + & - &
\cdots \cr - & + & - & + & \vdots \cr + & - & + & - & \vdots \cr
 \vdots  & \vdots  & \vdots  &  \vdots  & \vdots \cr
\end{bmatrix}
$$
\end{rem}
\begin{exa}
Find $$\det\begin{bmatrix} 1 & 2 & 3 \cr 4 & 5 & 6 \cr  7 & 8 & 9
\cr
\end{bmatrix}$$by expanding along the first row.
\end{exa}
\begin{solu}We have \begin{eqnarray*}\det A  & = & 1(-1)^{1 + 1}\det
\begin{bmatrix} 5 & 6 \cr 8 & 9 \end{bmatrix} + 2(-1)^{1 + 2}\det
\begin{bmatrix} 4 & 6 \cr 7 & 9 \end{bmatrix} + 3(-1)^{1 + 3}\det
\begin{bmatrix} 4 & 5 \cr 7 & 8 \end{bmatrix} \\ &  =  & 1(45 - 48) - 2(36
- 42) + 3(32 - 35 ) = 0. \end{eqnarray*}
\end{solu}

\begin{exa}
 Evaluate the {\em Vandermonde} determinant
 $$\det\begin{bmatrix}1 & 1 & 1 \cr a & b & c \cr a^2 & b^2 & c^2 \cr
 \end{bmatrix}.$$\end{exa}
 \begin{solu}$$\begin{array}{lll}\det\begin{bmatrix}1 & 1 & 1 \cr a & b & c \cr a^2 & b^2 & c^2 \cr \end{bmatrix}
 & = & \det\begin{bmatrix}1 & 0
 & 0 \cr a & b - a & c - a \cr a^2 & b^2 - a^2 & c^2 - a^2\cr \end{bmatrix} \\
 & = & \det \begin{bmatrix}b - a & c - a \cr b^2 - c^2 & c^2 - a^2 \end{bmatrix} \\ &
 = & (b - a)(c - a) \det\begin{bmatrix}1 & 1 \cr b + a & c + a \end{bmatrix} \\ &  = &
 (b - a)(c - a)(c - b).
 \end{array}$$
\end{solu}
 \begin{exa}
Evaluate the determinant
 $$\det A = \det\begin{bmatrix}1 & 2 & 3 & 4 & \cdots & 2000 \cr
 2 & 1 & 2 & 3 & \cdots & 1999 \cr 3 & 2 & 1 & 2 & \cdots & 1998
 \cr 4 & 3 & 2 & 1 & \cdots & 1997 \cr \cdots & \cdots & \cdots &
 \cdots & \cdots & \cdots \cr  2000 & 1999 & 1998 & 1997 & \cdots &
 1 \cr \end{bmatrix}.$$
 \end{exa}\begin{solu}Applying $R_n - R_{n + 1}\rightarrow R_n$ for $1
 \leq n \leq 1999,$ the determinant becomes
 $$\det\begin{bmatrix}-1 & 1 & 1 & 1 & \cdots  & 1 & 1 \cr
 -1 & -1 & 1 & 1 & \cdots & 1 & 1 \cr -1 & -1 & -1 & 1 & \cdots & 1
 & 1 \cr -1 & -1 & -1 & -1 & \cdots & 1 & 1 \cr \cdots & \cdots &
 \cdots & \cdots & \cdots & \cdots & \cdots \cr -1 & -1 & -1 & -1 &
 \cdots & -1 & 1 \cr 2000 & 1999 & 1998 & 1997 & \cdots  & 2 & 1
 \cr \end{bmatrix}.$$

 Applying now $C_n + C_{2000}\rightarrow C_n$ for $1 \leq n \leq
 1999,$ we obtain
 $$\det\begin{bmatrix}0 & 2 & 2 & 2 & \cdots  & 2 & 1 \cr
 0 & 0 & 2 & 2 & \cdots & 2 & 1 \cr 0 & 0 & 0 & 2 & \cdots  & 2 & 1
 \cr 0 & 0 & 0 & 0 & \cdots & 2 & 1 \cr \cdots & \cdots & \cdots &
 \cdots & \cdots & \cdots & \cdots \cr 0 & 0 & 0 & 0 & \cdots & 0 &
 1 \cr 2001 & 2000 & 1999 & 1998 & \cdots  & 3 & 1 \cr \end{bmatrix}.$$ This
 last determinant we expand along the first column. We have
 $$2001\det\begin{bmatrix}2 & 2 & 2 & \cdots  & 2 & 1 \cr
 0 & 2 & 2 & \cdots & 2 & 1 \cr 0 & 0 & 2 & \cdots  & 2 & 1 \cr  0
 & 0 & 0 & \cdots & 2 & 1 \cr  \cdots & \cdots & \cdots & \cdots &
 \cdots & \cdots \cr  0 & 0 & 0 & \cdots & 0 & 1 \cr \end{bmatrix} =
 2001(2^{1998}).$$
\end{solu}
\begin{df}Let $A\in\mat{n\times n}{ \bbR}$. The \negrito{classical adjoint} or
\negrito{adjugate} of $A$ is the $n\times n$ matrix $\adj{A}$ whose
entries are given by \index{matrix!adjoint}
$$[\adj{A}]_{ij} = (-1)^{i + j} \det A_{ji},
$$where $A_{ji}$ is the $ji$-th minor of $A$.
\end{df}
\begin{thm}Let $A\in\mat{n\times n}{ \bbR}$.
Then $$(\adj{A})A = A(\adj{A}) = (\det A){\textbf  I}_n.    $$
\end{thm}
\begin{proof}
We have $$ \begin{array}{lll} [A(\adj{A})]_{ij} & = & \sum_{k=1}
^n a_{ik}[\adj{A}]_{kj} \\
& = & \sum _{k=1} ^n a_{ik}(-1)^{i + k}\det A_{jk}.
\end{array}$$Now, this last sum is $\det A$ if $i = j$ by virtue of Theorem \ref{thm:laplace_expansion}. If $i\neq
j$ it is $0$, since then the $j$-th row is identical to the $i$-th
row and this determinant is $0_{\bbR}$ by virtue of Lemma
\ref{lem:determinant_with_two_identical_rows}. Thus on the diagonal
entries we get $\det A$ and the off-diagonal entries are $0_{\bbR
}$. This proves the theorem.
\end{proof}
The next corollary follows immediately.
\begin{cor}\label{cor:inverse_via_adjoint}Let $A\in\mat{n\times n}{ \bbR}$.
Then $A$ is invertible if and only $\det A \neq 0_{\bbR}$ and
$$A^{-1} = \dfrac{\adj{A}}{\det A}.
$$
\end{cor}

\section*{\psframebox{Homework}}

\begin{pro}
Find $$\det\begin{bmatrix} 1 & 2 & 3 \cr 4 & 5 & 6 \cr  7 & 8 & 9
\cr
\end{bmatrix}$$by expanding along the second column.
\begin{answer} We have
\begin{eqnarray*}\det A  & = & 2(-1)^{1 + 2}\det
\begin{bmatrix} 4 & 6 \cr 7 & 9 \end{bmatrix} + 5(-1)^{2 + 2}\det
\begin{bmatrix} 1 & 3 \cr 7 & 9 \end{bmatrix} + 8(-1)^{2 + 3}\det
\begin{bmatrix} 1 & 3 \cr 4 & 6 \end{bmatrix} \\ &  =  & -2(36 - 42) +
5(9 - 21) - 8(6 - 12) = 0. \end{eqnarray*}
\end{answer}
\end{pro}
\begin{pro}
Prove that $\det \begin{bmatrix} a & b & c \cr c &a& b \cr b &c & a
\cr
\end{bmatrix} = a^3+b^3+c^3-3abc. $ This type of matrix is called a
{ circulant matrix}.
\begin{answer} Simply expand along the first row$$a\det\begin{bmatrix} a & b
\cr c & a
\end{bmatrix} - b\det\begin{bmatrix}
c & b \cr b & a
\end{bmatrix} + c\det\begin{bmatrix}
c & a \cr b & c\end{bmatrix} = a(a^2 -bc ) - b(ca - b^2) + c(c^2 -
ab) = a^3+b^3+c^3-3abc .
$$
\end{answer}

\end{pro}
\begin{pro}Compute the determinant
$$\det \begin{bmatrix}
     1 & 0 & -1 & 1 \cr
     2 & 0 & 0 & 1 \cr
     666 & -3  & -1 & 1000000 \cr
     1 & 0 & 0 &1 \cr
\end{bmatrix}.$$
\begin{answer} Since the second column has three $0$'s, it is advantageous
to expand along it, and thus we are reduced to calculate
$$-3(-1)^{3 + 2} \det \begin{bmatrix}
                1 & -1 & 1 \cr
                2 & 0 & 1 \cr
                1 & 0 & 1 \cr
\end{bmatrix}$$
Expanding this last determinant along the second column, the
original determinant is thus
$$-3(-1)^{3 + 2}(-1)(-1)^{1 + 2}\det \begin{bmatrix}
            2 & 1 \cr
            1 & 1 \cr
\end{bmatrix} = -3(-1)(-1)(-1)(1) = 3.$$
\end{answer}
\end{pro}
\begin{pro}
Prove that
$$\det \begin{bmatrix} x+a & b & c  \cr a & x+b & c \cr a & b & x+c
\cr
\end{bmatrix} = x^2(x+a+b+c). $$
\end{pro}
\begin{pro}
If
$$ \det\begin{bmatrix} 1 & 1 & 1 & 1 \cr x & a & 0 & 0 \cr
x & 0 & b & 0 \cr x & 0 & 0 & c\cr \end{bmatrix}  = 0, $$ and
$xabc\neq 0,$ prove that $$ \frac{1}{x} = \frac{1}{a} + \frac{1}{b}
+ \frac{1}{c}.  $$ \begin{answer} Expanding along the first column,
$$\begin{array}{lll} 0 & = &  \det\begin{bmatrix} 1 & 1 & 1 & 1 \cr x & a & 0 & 0 \cr
x & 0 & b & 0 \cr x & 0 & 0 & c\cr \end{bmatrix} \\
& = & \det\begin{bmatrix}a & 0 & 0 \cr 0 & b & 0 \cr 0 & 0 & c \cr
\end{bmatrix} - x\det\begin{bmatrix} 1 & 1 & 1 \cr 0 & b & 0 \cr 0
& 0 & c \cr \end{bmatrix} \\ & & \qquad  + x\det\begin{bmatrix}1 &
1 & 1 \cr a & 0 & 0 \cr 0 & 0 & c \cr  \end{bmatrix}
- x\det\begin{bmatrix} 1 & 1 & 1 \cr a & 0 & 0 \cr  0 & b & 0 \cr \end{bmatrix} \\
& = & xabc - xbc + x\det\begin{bmatrix}1 & 1 & 1 \cr a & 0 & 0 \cr
0 & 0 & c \cr  \end{bmatrix} - x\det\begin{bmatrix} 1 & 1 & 1 \cr
a & 0 & 0 \cr  0 & b & 0 \cr \end{bmatrix}.\\ \end{array}$$
Expanding these  last  two determinants along the third row,
$$\begin{array}{lll} 0 & = & abc - xbc
+ x\det\begin{bmatrix}1 & 1 & 1 \cr a & 0 & 0 \cr 0 & 0 & c \cr
\end{bmatrix}
- x\det\begin{bmatrix} 1 & 1 & 1 \cr a & a & 0 \cr  0 & b & 0 \cr \end{bmatrix}\\
& = &   abc - xbc + xc\det \begin{bmatrix} 1 & 1 \cr  a & 0 \cr\end{bmatrix}  + xb\det \begin{bmatrix} 1 & 1 \cr  a & 0 \cr\end{bmatrix} \\
& = & abc - xbc - xca -xab.
     \end{array}$$It follows that $$abc = x(bc + ab + ca), $$whence$$\frac{1}{x} = \frac{bc + ab + ca}{abc} = \frac{1}{a} +\frac{1}{b} + \frac{1}{c},  $$as wanted.

\end{answer}
\end{pro}
\begin{pro}
Consider the matrix $$A = \begin{bmatrix}a & -b & -c & -d \cr b  & a
& d & -c \cr c & -d & a & b \cr d & c & -b & a \cr  \end{bmatrix}.$$
\begin{dingautolist}{202}
\item  Compute $A^TA$. \item   Use the above to prove that $$
\det A = (a^2+b^2+c^2+d^2)^2.$$

\end{dingautolist}
\end{pro}
\begin{pro}
Prove that $$\det \begin{bmatrix}0 & a& b & 0 \cr a & 0 & b & 0
\cr 0 & a & 0 & b \cr 1 & 1 & 1& 1\cr
\end{bmatrix} = 2ab(a-b).
$$
\begin{answer} Expanding along the first row the determinant equals
$$\begin{array}{lll}-a\det\begin{bmatrix} a & b & 0 \cr
0 & 0 & b \cr 1 & 1 & 1 \cr \end{bmatrix} + b\det\begin{bmatrix} a
& 0 & 0 \cr 0 & a & b \cr 1 & 1 & 1 \cr\end{bmatrix}  & = &  ab
\det\begin{bmatrix} a & b \cr 1 & 1  \cr
\end{bmatrix} + ab\det\begin{bmatrix} a & b \cr 1 & 1  \cr
\end{bmatrix}  \\ & =  & 2ab (a - b), \end{array}$$as wanted.
\end{answer}
\end{pro}
\begin{pro}
Demonstrate that  $$\det \begin{bmatrix} a & 0 & b & 0 \cr 0 & a &
0 & b \cr c & 0 & d & 0 \cr 0 & c & 0 & d\cr
\end{bmatrix} = (ad - bc)^2.
$$
\begin{answer} Expanding along the first row, the determinant equals
$$ a\det\begin{bmatrix} a & 0 & b \cr 0 & d & 0 \cr c & 0 & d \cr  \end{bmatrix}
+ b \det\begin{bmatrix} 0 & a & b \cr c & 0 & 0 \cr 0 & c & d \cr
\end{bmatrix}.
$$Expanding the resulting two determinants along the second row,
we obtain $$ad\det\begin{bmatrix}a & b \cr c & d\cr \end{bmatrix}
+ b(-c)\det\begin{bmatrix}a & b \cr c & d\cr  \end{bmatrix} =
ad(ad - bc) - bc(ad - bc) = (ad - bc)^2,
$$as wanted.
\end{answer}
\end{pro}
\begin{pro}
Use induction to shew that
$$ \det\begin{bmatrix} 1 & 1 & 1 & \cdots & 1 &
1 \cr  1& 0 & 0 & \vdots & 0  & 0 \cr  0 & 1 & 0 & \cdots & 0  & 0
\cr 0 & 0 & 1& \cdots & 0  & 0 \cr \vdots & \vdots & \cdots &
\vdots & \vdots \cr 0 & 0 & 0 & \cdots& 1 & 0 \cr
\end{bmatrix} = (-1)^{n + 1}.$$
\label{exa:determinant_bunch_of_ones}\begin{answer} For $n = 1$ we
have $\det (1) = 1 = (-1)^{1 + 1}$. For $n = 2$ we have
$$\det \begin{bmatrix} 1 & 1 \cr 1 & 0 \cr
\end{bmatrix} = -1 = (-1)^{2 + 1}.
$$Assume that the result is true for $n -1$. Expanding the
determinant along the first column
$$ \begin{array}{lll}\det\begin{bmatrix} 1 & 1 & 1 & \cdots & 1 &
1 \cr  1& 0 & 0 & \vdots & 0  & 0 \cr  0 & 1 & 0 & \cdots & 0  & 0
\cr 0 & 0 & 1& \cdots & 0  & 0 \cr \vdots & \vdots & \cdots &
\vdots & \vdots \cr 0 & 0 & 0 & \cdots & 1 & 0 \cr
\end{bmatrix} &  = & 1\det\begin{bmatrix}  0 & 0 & \vdots & 0  & 0 \cr   1 & 0 & \cdots & 0  & 0
\cr  0 & 1& \cdots & 0  & 0 \cr  \vdots & \cdots & \vdots & \vdots
\cr  0 & 0 & \cdots & 1 & 0 \cr
\end{bmatrix} \\ & & \qquad - 1 \det\begin{bmatrix}  1 & 1 & \cdots & 1 &
1 \cr  1& 0 & \cdots & \vdots & 0   \cr  0 & 1 & \cdots & \cdots &
0 \cr 0 & 0 & \cdots & \cdots & 0   \cr \vdots & \vdots & \cdots &
\vdots \cr 0 & 0 & \cdots & 1 & 0 \cr
\end{bmatrix}\\
& = & 1(0) - (1)(-1)^{n}\\
& = &  (-1)^{n + 1},\\ \end{array}$$giving the result.
\end{answer}
\end{pro}
\begin{pro}
Let $$A = \begin{bmatrix} 1 & n & n & n & \cdots & n \cr n & 2 & n
& n & \vdots & n \cr n & n & 3 & n & \cdots & n \cr n & n & n & 4
& \cdots & n \cr \vdots & \vdots & \vdots & \cdots & \vdots \cr n
& n & n & n & n & n \cr
\end{bmatrix},
$$that is, $A\in\mat{n\times n}{\bbR}, A = [a_{ij}]$ is a matrix such that $a_{kk} = k$
and $a_{ij} = n$ when $i \neq j$. Find $\det A$. \begin{answer}
Perform $C_k - C_1 \rightarrow C_k$ for $k \in [2; n]$. Observe that
these operations do not affect the value of the determinant. Then
$$\det A = \det\begin{bmatrix} 1 & n - 1 & n - 1 & n - 1 & \cdots & n - 1 \cr n & 2 - n &
0 & 0 & \vdots & 0 \cr n & 0 & 3 - n & 0 & \cdots & 0 \cr n & 0 &
0 & 4-n & \cdots & 0 \cr \vdots & \vdots & \vdots & \cdots &
\vdots \cr n & 0 & 0 & 0 & 0 & 0 \cr
\end{bmatrix}.
$$Expand this last determinant along the $n$-th row, obtaining,
$$\begin{array}{lll}\det A & = &  (-1)^{1 + n}n\det\begin{bmatrix}  n - 1 & n - 1 & n - 1 & \cdots & n - 1 & n - 1 \cr  2 - n &
0 & 0 & \vdots & 0  & 0 \cr  0 & 3 - n & 0 & \cdots & 0  & 0 \cr 0
& 0 & 4-n & \cdots & 0  & 0 \cr  \vdots & \vdots & \cdots & \vdots
& \vdots \cr 0 & 0 & 0 & \cdots & -1 & 0  \cr
\end{bmatrix} \\
& = & (-1)^{1 + n}n(n - 1)(2 - n)(3 - n)\\ & & \qquad \cdots
(-2)(-1) \det\begin{bmatrix}  1 & 1 & 1 & \cdots & 1 & 1 \cr  1& 0
& 0 & \vdots & 0  & 0 \cr  0 & 1 & 0 & \cdots & 0  & 0 \cr 0 & 0 &
1& \cdots & 0  & 0 \cr  \vdots & \vdots & \cdots & \vdots & \vdots
\cr 0 & 0 & 0 & \cdots & 1 & 0  \cr
\end{bmatrix} \\
& = & -(n!)\det\begin{bmatrix}  1 & 1 & 1 & \cdots & 1 & 1 \cr  1&
0 & 0 & \vdots & 0  & 0 \cr  0 & 1 & 0 & \cdots & 0  & 0 \cr 0 & 0
& 1& \cdots & 0  & 0 \cr  \vdots & \vdots & \cdots & \vdots &
\vdots \cr 0 & 0 & 0 & \cdots & 1 & 0  \cr
\end{bmatrix} \\
& = & -(n!)(-1)^{n} \\
& = & (-1)^{n + 1}n!,
\end{array}$$
upon using the result of problem
\ref{exa:determinant_bunch_of_ones}.
\end{answer}
\end{pro}
\begin{pro}
Let $n\in\BBN , n > 1$ be an odd integer. Recall that the binomial
coefficients $\binom{n}{k}$ satisfy $\binom{n}{n} = \binom{n}{0} =
1$ and that for $1 \leq k \leq n$,
$$\binom{n}{k} = \binom{n - 1}{k - 1} + \binom{n - 1}{k}.  $$Prove that
$$ \det\begin{bmatrix} 1 & \binom{n}{1} & \binom{n}{2} & \cdots & \binom{n}{n - 1} & 1 \cr
1 & 1 & \binom{n}{1} & \cdots & \binom{n}{n - 2} & \binom{n}{n -
1} \cr \binom{n}{n - 1} & 1 & 1 & \cdots & \binom{n}{n - 3} &
\binom{n}{n - 2} \cr
 \cdots  &  \cdots  &  \cdots  & \cdots &  \cdots  &  \cdots  \cr
 \binom{n}{1} & \binom{n}{2} & \binom{n}{3} & \cdots & 1 & 1 \cr
 \end{bmatrix}  = (1 + (-1)^n)^n.$$
\begin{answer} Recall
that $\binom{n}{k} = \binom{n}{n - k}$, $$ \sum _{k = 0} ^n
\binom{n}{k} = 2^n  $$ and$$ \sum _{k = 0} ^n (-1)^{k}\binom{n}{k}
= 0, \ \ \ \ {\textrm  if}\ \ n > 0. $$ Assume that $n$ is odd. Observe
that then there are $n + 1$  (an even number) of columns and that
on the same row, $\binom{n}{k}$ is on a column of opposite parity
to that of $\binom{n}{n - k}$. By performing  $C_1 - C_2 + C_3 -
C_4 + \cdots + C_n - C_{n + 1} \rightarrow C_1$, the first column
becomes all $0$'s, whence the determinant if $0$ if $n$ is odd.
\end{answer}
\end{pro}
\begin{pro}
Let $A\in\gl{n}{ \bbR}$, $n>1$. Prove that $\det (\adj{A}) = (\det
A)^{n-1}$.
\end{pro}
\begin{pro} Let $(A, B, S)\in(\gl{n}{ \bbR})^3$. Prove that\begin{dingautolist}{202} \item $
\adj{\adj{A}} = (\det A)^{n-2}A$.

\item $\adj{AB} = \adj{A}\adj{B}$. \item $\adj{SAS^{-1}} = S(\adj{
A})S^{-1}$.
\end{dingautolist}\end{pro}
\begin{pro} If $A\in\gl{2}{\bbR}$, ,  and let  $k$ be
a positive integer. Prove that $\det (\underbrace{\mathrm{adj}
\cdots \mathrm{adj}}_{k}(A)) = \det A$.
\end{pro}
\begin{pro}
Find the determinant
$$\det \begin{bmatrix}  (b+c)^2 &  ab & ac \cr
ab & (a+c)^2 & bc\cr ac & bc & (a+b)^2 \cr
\end{bmatrix}  $$
{\em by hand, making explicit all your calculations.}
\begin{answer}
I will prove that
$$\det \begin{bmatrix}  (b+c)^2 &  ab & ac \cr
ab & (a+c)^2 & bc\cr ac & bc & (a+b)^2 \cr
\end{bmatrix} = 2abc(a+b+c)^3. $$
Using permissible row and column operations,
$$\begin{array}{lll}
\det \begin{bmatrix}  (b+c)^2 &  ab & ac \cr ab & (a+c)^2 & bc\cr ac
& bc & (a+b)^2 \cr
\end{bmatrix} & = & \det \begin{bmatrix}  b^2+2bc+c^2 &  ab & ac \cr ab & a^2+2ca+c^2 & bc\cr ac
& bc & a^2+2ab+b^2 \cr
\end{bmatrix}\\
& = \grstep{C_1+C_2+C_3\to C_1} & \det
\begin{bmatrix} b^2+2bc+c^2+ab+ac & ab & ac \cr ab+a^2+2ca+c^2+bc & a^2+2ca+c^2 & bc\cr
ac+bc+a^2+2ab+b^2 & bc & a^2+2ab+b^2 \cr
\end{bmatrix}\\
& =  & \det
\begin{bmatrix} (b+c)(a+b+c) & ab & ac \cr (a+c)(a+b+c) & a^2+2ca+c^2 & bc\cr
(a+b)(a+b+c) & bc & a^2+2ab+b^2 \cr
\end{bmatrix}\\
\end{array}$$
Pulling out a factor, the above equals
$$
 (a+b+c)\det
\begin{bmatrix} b+c & ab & ac \cr a+c & a^2+2ca+c^2 & bc\cr
a+b & bc & a^2+2ab+b^2 \cr
\end{bmatrix}
$$ and performing $R_1+R_2+R_3\to R_1$, this is  $$ (a+b+c)\det
\begin{bmatrix} 2a+2b+2c & ab+a^2+2ca+c^2+bc & ac+bc+ a^2+2ab+b^2 \cr a+c & a^2+2ca+c^2 & bc\cr
a+b & bc & a^2+2ab+b^2 \cr
\end{bmatrix}$$
Factoring this is
$$ (a+b+c)\det
\begin{bmatrix} 2(a+b+c) & (a+c)(a+b+c) & (a+b)(a+b+c) \cr a+c & a^2+2ca+c^2 & bc\cr
a+b & bc & a^2+2ab+b^2 \cr
\end{bmatrix},$$which in turn is  $$(a+b+c)^2\det
\begin{bmatrix} 2 & a+c & a+b \cr a+c & a^2+2ca+c^2 & bc\cr
a+b & bc & a^2+2ab+b^2 \cr
\end{bmatrix}$$
Performing $C_2-(a+c)C_1\to C_2$ and $C_3-(a+b)C_1\to C_3$ we obtain
 $$(a+b+c)^2\det
\begin{bmatrix} 2 & -a-c & -a-b \cr a+c & 0 & -a^2-ab-ac\cr
a+b &-a^2- ab-ac & 0 \cr
\end{bmatrix}$$
This last matrix we will expand by the second column, obtaining that
the original determinant is thus
$$ (a+b+c)^2\left((a+c)\det\begin{bmatrix}a+c & -a^2-ab-ac \cr  a+b & 0  \end{bmatrix} +(a^2+ab+ac)\det\begin{bmatrix} 2 & -a-b \cr a+c & -a^2-ab-ac \end{bmatrix}\right)  $$
This simplifies to
$$\begin{array}{lll}
(a+b+c)^2\left((a+c)(a+b)(a^2+ab+ac\right.)\\
\qquad \left.+(a^2+ab+ac)(-a^2-ab-ac+bc)\right) & = &
a(a+b+c)^3((a+c)(a+b)-a^2-ab-ac+bc)\\ & = &
2abc(a+b+c)^3,\end{array}
$$ as claimed.




\end{answer}


\end{pro}

\begin{pro}
The matrix  $$  \begin{bmatrix} a & b & c & d \cr d & a & b & c \cr c & d & a & b \cr b & c & d & a\cr         \end{bmatrix}
$$ is known as a \negrito{circulant matrix}. Prove that its determinant is $(a+b+c+d)(a-b+c-d)((a-c)^2+(b-d)^2)$.
\begin{answer}
 We have

 \begin{eqnarray*}\det \begin{bmatrix} a & b & c & d \cr d & a & b
& c \cr c & d & a & b \cr b & c & d & a\cr         \end{bmatrix} &
\grstep[=]{R_1+R_2+R_3+R_4\to R_1} & \det \begin{bmatrix} a+b+c+d &
a+b+c+d & a+b+c+d & a+b+c+d \cr d & a & b & c \cr c & d & a & b \cr
b & c & d & a\cr
\end{bmatrix}\\
& = & (a+b+c+d)\det \begin{bmatrix} 1 & 1 & 1 & 1 \cr d & a & b & c
\cr c & d & a & b \cr b & c & d & a\cr         \end{bmatrix}\\
& \grstep[=]{C_4-C_3+C_2-C_1\to C_4} & (a+b+c+d)\begin{bmatrix} 1 &
1 & 1 & 0\cr d & a & b & c-b+a-d \cr c & d & a & b-a+d-c \cr b & c &
d & a-d+c-b\cr
\end{bmatrix}\\
& = &  (a+b+c+d)(a-b+c-d)\begin{bmatrix} 1 & 1 & 1 & 0\cr d & a & b
& 1\cr c & d & a & -1 \cr b & c & d & 1\cr
\end{bmatrix}\\
& \grstep[=]{R_2+R_3\to R_2,\ R_4+R_3\to R_4} &
(a+b+c+d)(a-b+c-d)\begin{bmatrix} 1 & 1 & 1 & 0\cr d+c & a+d & b+a &
0\cr c & d & a & -1 \cr b+c & c+d & a+d & 0\cr
\end{bmatrix}\\
&= & (a+b+c+d)(a-b+c-d)\begin{bmatrix} 1 & 1 & 1 \cr d+c & a+d & b+a
\cr  b+c & c+d & a+d \cr
\end{bmatrix}\\
& \grstep[=]{C_1-C_3\to C_1, \ C_2-C_3\to C_2} &
(a+b+c+d)(a-b+c-d)\begin{bmatrix} 0 & 0 & 1 \cr d+c-b-a & d-b & b+a
\cr b+c-a-d & c-a
 & a+d \cr
\end{bmatrix}\\
& = & (a+b+c+d)(a-b+c-d)\begin{bmatrix}  d+c-b-a & d-b \cr b+c-a-d &
c-a
 \cr
\end{bmatrix}\\
& = & (a+b+c+d)(a-b+c-d) (d+c-b-a)(c-a)-(d-b)(b+c-a-d)
\\
& = & (a+b+c+d)(a-b+c-d)\\
& & \qquad
((c-a)(c-a)+(c-a)(d-b)-(d-b)(c-a)-(d-b)(b-d))
\\
& = & (a+b+c+d)(a-b+c-d)((a-c)^2+(b-d)^2).
\end{eqnarray*}
\begin{rem}
Since $$ (a-c)^2+(b-d)^2=(a-c+i(b-d))(a-c-i(b-d)), $$ the
above determinant is then
$$ (a+b+c+d)(a-b+c-d)(a+ib-c-id)(a-ib-c+id). $$
Generalisations of this determinant are possible using roots of
unity.
\end{rem}
\end{answer}
\end{pro}


\section{Determinants and Linear Systems}
\begin{thm}\label{thm:systems_determinants_invertibility}Let
$A\in\mat{n\times n}{ \bbR}$. The following are all equivalent
\begin{dingautolist}{202}
\item \label{thm:sys_det_1} $\det A \neq 0_{\bbR}$. \item
\label{thm:sys_det_2}$A$ is invertible. \item\label{thm:sys_det_3}
There exists a unique solution $X\in\mat{n\times 1}{ \bbR}$ to the
equation $AX = Y$. \item  \label{thm:sys_det_4}If $AX = {\textbf 
0}_{n\times 1}$ then $X = {\textbf  0}_{n\times 1}$.
\end{dingautolist}\end{thm}
\begin{proof} We prove the implications in sequence: \\
$\ref{thm:sys_det_1} \implies \ref{thm:sys_det_2}$: follows from
Corollary \ref{cor:inverse_via_adjoint}\\ $\ref{thm:sys_det_2}
\implies \ref{thm:sys_det_3}$: If $A$ is invertible and $AX = Y$ then $X = A^{-1}Y$ is the unique solution of this equation.\\
$\ref{thm:sys_det_3} \implies \ref{thm:sys_det_4}$: follows by putting $Y = {\textbf  0}_{n\times 1}$\\
$\ref{thm:sys_det_4} \implies \ref{thm:sys_det_1}$: Let $R$ be the
row echelon form of $A$. Since $RX = {\textbf  0}_{n\times 1}$ has only
$X={\textbf  0}_{n\times 1}$ as a solution, every entry on the diagonal
of $R$ must be non-zero, $R$ must be triangular, and hence $\det R
\neq 0_{\bbR}$. Since $A = PR$ where $P$ is an invertible $n\times
n$ matrix,
we deduce that $\det A = \det P\det R  \neq 0_{\bbR}$.\\
\end{proof}
The contrapositive form of the implications \ref{thm:sys_det_1}
and \ref{thm:sys_det_4}  will be used later. Here it is for future
reference.
\begin{cor}\label{cor:0determinant_non_null_kernel}
Let $A\in\mat{n\times n}{ \bbR}$. If there is $X\neq {\textbf 
0}_{n\times 1}$ such that $AX = {\textbf  0}_{n\times 1}$ then $\det A =
0_{\bbR}$.
\end{cor}

\section*{\psframebox{Homework}}
\begin{pro}
For which $a$ is the matrix $\begin{bmatrix}-1 & 1 & 1 \cr 1 & a & 1
\cr 1 & 1 & a \cr \end{bmatrix}$ singular (non-invertible)?
\end{pro}
